\documentclass{article}
\usepackage[utf8]{inputenc}
\usepackage[a4paper, margin=1in]{geometry}
\usepackage{siunitx}
\usepackage{amsmath}
\usepackage{enumitem}
\usepackage{esdiff}
\usepackage{pgfplots}
\usepackage{listings}
\usepackage{xcolor}

\pgfplotsset{compat=1.18, width=10cm}

\tolerance=1
\emergencystretch=\maxdimen
\hyphenpenalty=10000
\hbadness=10000

\sisetup{
    input-ignore={.},
    output-decimal-marker={,},
    group-minimum-digits=4,
    group-separator={.},
    group-digits=integer
}

\definecolor{darkgray}{rgb}{.4,.4,.4}

\lstdefinestyle{code}{
    aboveskip={1.3\baselineskip},
    basicstyle=\normalsize\ttfamily\linespread{4},
    breaklines=false,
    columns=fullflexible,
    commentstyle=\color[rgb]{0.127,0.427,0.514}\ttfamily\itshape,
    escapechar=@,
    extendedchars=true,
    frame=single,
    identifierstyle=\color{black},
    inputencoding=latin1,
    keywordstyle=\color[HTML]{228B22}\bfseries,
    ndkeywordstyle=\color[HTML]{228B22}\bfseries,
    numbers=left,
    numberstyle=\normalsize,
    prebreak = \raisebox{0ex}[0ex][0ex]{\ensuremath{\hookleftarrow}},
    stringstyle=\color[rgb]{0.639,0.082,0.082}\ttfamily,
    upquote=true,
    showstringspaces=false,
    xleftmargin=5ex,
    aboveskip=5pt
}

\newcommand{\penyelesaian}{\textbf{Penyelesaian: }}

\title{\textbf{Komputasi Numerik: Tugas 5}}
\author{Kelompok 15}
\date{}

\begin{document}

\maketitle

\begin{enumerate}
    \item Carilah $\int f(x) dx$ dari data-data berikut dengan batas $x=1$ sampai $x=7$ menggunakan integrasi Trapezoida, Simpson 1/3, dan Simpson 3/8 jika diketahui data-data berikut: \\
    \begin{tabular}{ c c c c c c c c }
        $x$ & $\num{1,0}$ & $\num{2,0}$ & $\num{3,0}$ & $\num{4,0}$ & $\num{5,0}$ & $\num{6,0}$ & $\num{7,0}$\\
        $f(x)$ & \num{1,8287} & \num{5,6575} & \num{11,4862} & \num{19,3149} & \num{29,1437} & \num{40,9724} & \num{54,8011} \\
    \end{tabular}
    \begin{tabular}{ c c c c c c c c }
        $x$ & $\num{1,0}$ & $\num{2,0}$ & $\num{3,0}$ & $\num{4,0}$ & $\num{5,0}$ & $\num{6,0}$ & $\num{7,0}$\\
        $f(x)$ & \num{2,1353} & \num{6,2707} & \num{12,4060} & \num{20,5413} & \num{30,6767} & \num{42,8120} & \num{56,9473} \\
    \end{tabular}
    \begin{tabular}{ c c c c c c c c }
        $x$ & $\num{1,0}$ & $\num{2,0}$ & $\num{3,0}$ & $\num{4,0}$ & $\num{5,0}$ & $\num{6,0}$ & $\num{7,0}$\\
        $f(x)$ & \num{1,8419} & \num{5,6838} & \num{11,5257} & \num{19,3676} & \num{29,2095} & \num{41,0514} & \num{54,8933} \\
    \end{tabular} \\
    \penyelesaian
    \begin{itemize}
        \item Data pertama: \\
        \begin{itemize}
            \item Metode Trapezoida \\
            Dengan $\Delta x = 1,$
            \begin{align*}
                I &= \int_{1}^{7} f(x) dx \\
                &\approx \frac{\Delta x}{2} \left [f(1) + f(7) + 2 \left [f(2) + f(3) + f(4) + f(5) + f(6)\right ] \right ] \\
                &\approx \frac{1}{2} \left [\num{1,8287} + \num{54,8011} + 2 \left [\num{5,6575} + \num{11,4862} + \num{19,3149} + \num{29,1437} + \num{40,9724}\right ] \right ] \\
                &\approx \frac{1}{2} \left [\num{1,8287} + \num{54,8011} + 2 \left [\num{106,5747}\right ] \right ] \\
                &\approx \frac{1}{2} (\num{269,7792}) \\
                &\approx \num{134,8896}
            \end{align*}
            \item Metode Simpson 1/3 \\
            Dengan $\Delta x = 1,$
            \begin{align*}
                I &= \int_{1}^{7} f(x) dx \\
                &\approx \frac{\Delta x}{3} [f(1) + 4[f(2) + f(4) + f(6)] + 2[f(3) + f(5)] + f(7)] \\
                &\approx \frac{1}{3} [\num{1,8287} + 4[\num{5,6575} + \num{19,3149} + \num{40,9724}] + 2[\num{11,4862} + \num{29,1437}] + \num{54,8011}] \\
                &\approx \frac{1}{3} [\num{1,8287} + 4[\num{65,9448}] + 2[\num{40,6299}] + \num{54,8011}] \\ 
                &\approx \frac{1}{3} (\num{401,6688}) \\
                &\approx \num{133,8896}
            \end{align*}
            \item Metode Simpson 3/8 \\
            Dengan $\Delta x = 1,$
            \begin{align*}
                I &= \int_{1}^{7} f(x) dx \\
                &\approx \frac{3\Delta x}{8} [f(1) + 3[f(2) + f(3) + f(5) + f(6)] + 3[f(4)] + f(7)] \\
                &\approx \frac{3}{8} [\num{1,8287} + 3(\num{5,6575} + \num{11,4862} + \num{29,1437} + \num{40,9724}) + 2(\num{19,3149}) + \num{54,8011}] \\
                &\approx \frac{3}{8} [\num{1,8287} + 3(\num{87,2598}) + \num{38,6298} + \num{54,8011}] \\
                &\approx \frac{3}{8} [\num{1,8287} + \num{261,7794} + \num{38,6298} + \num{54,8011}] \\
                &\approx \frac{3}{8} (\num{357,0390}) \\
                &\approx \num{133,8896}
            \end{align*}

        \end{itemize}

        \item Data kedua: \\
        \subsection*{Data}
        \begin{center}
        \begin{tabular}{|c|ccccccc|}
        \hline
        $x$    & 1,0     & 2,0     & 3,0     & 4,0     & 5,0     & 6,0     & 7,0     \\ \hline
        $f(x)$ & 2,1353  & 6,2707  & 12,4060 & 20,5413 & 30,6767 & 42,8120 & 56,9473 \\ \hline
        \end{tabular}
        \end{center}

        \textbf{Asumsi:} \( I_{\text{eksak}} = 140 \) .

        \subsection*{1. Metode Trapezoida}
        \subsubsection*{a. Trapezoida (6 pias, $h=1$)}
        \[
        I = \frac{h}{2} \left[ f(1) + 2\sum_{i=2}^{6} f(x_i) + f(7) \right] = \boxed{142,2480}
        \]
        \[
        \text{Error Relatif} = \left| \frac{142,2480 - 140}{140} \right| \times 100\% = \boxed{1,61\%}
        \]

        \subsubsection*{b. Trapezoida Koreksi Ujung}
        \[
        f'(1) \approx \frac{-f(3) + 4f(2) - 3f(1)}{2h} = 2,1353, \quad f'(7) \approx \frac{3f(7) - 4f(6) + f(5)}{2h} = 14,8120
        \]
        \[
        \text{Koreksi} = -\frac{h^2}{12} \left[ f'(7) - f'(1) \right] = -1,0564
        \]
        \[
        I_{\text{final}} = 142,2480 - 1,0564 = \boxed{141,1916}
        \]
        \[
        \text{Error Relatif} = \left| \frac{141,1916 - 140}{140} \right| \times 100\% = \boxed{0,85\%}
        \]

        \subsection*{2. Metode Simpson 1/3}
        \subsubsection*{a. 2 Pias ($n=2$, $\Delta x=3$)}
        \[
        I = \frac{3}{3} \left[ f(1) + 4f(4) + f(7) \right] = \boxed{140,6478}
        \]
        \[
        \text{Error Relatif} = \left| \frac{140,6478 - 140}{140} \right| \times 100\% = \boxed{0,46\%}
        \]

        \subsubsection*{b. 6 Pias ($n=6$, $\Delta x=1$)}
        \[
        I = \frac{1}{3} \left[ f(1) + 4(f(2)+f(4)+f(6)) + 2(f(3)+f(5)) + f(7) \right] = \boxed{141,2480}
        \]
        \[
        \text{Error Relatif} = \left| \frac{141,2480 - 140}{140} \right| \times 100\% = \boxed{0,89\%}
        \]

        \subsection*{3. Metode Simpson 3/8}
        \subsubsection*{a. 3 Pias ($n=3$, $\Delta x=2$)}
        \[
        I = \frac{6}{8} \left[ f(1) + 3f(3) + 3f(5) + f(7) \right] = \boxed{119,0333}
        \]
        \[
        \text{Error Relatif} = \left| \frac{119,0333 - 140}{140} \right| \times 100\% = \boxed{14,98\%}
        \]

        \subsubsection*{b. 6 Pias ($n=6$, $\Delta x=1$)}
        \[
        I = \frac{3}{8} \left[ f(1) + 3(f(2)+f(3)+f(5)+f(6)) + 2f(4) + f(7) \right] = \boxed{158,2555}
        \]
        \[
        \text{Error Relatif} = \left| \frac{158,2555 - 140}{140} \right| \times 100\% = \boxed{13,04\%}
        \]

        \subsection*{Hasil dan Error Relatif}
        \begin{center}
        \begin{tabular}{|l|c|c|}
        \hline
        \textbf{Metode} & \textbf{Hasil Integral} & \textbf{Error Relatif} \\ \hline
        Trapezoida Standar (6 pias) & 142,2480 & 1,61\% \\ \hline
        Trapezoida + Koreksi Ujung & 141,1916 & 0,85\% \\ \hline
        Simpson 1/3 (2 pias) & 140,6478 & 0,46\% \\ \hline
        Simpson 1/3 (6 pias) & 141,2480 & 0,89\% \\ \hline
        Simpson 3/8 (3 pias) & 119,0333 & 14,98\% \\ \hline
        Simpson 3/8 (6 pias) & 158,2555 & 13,04\% \\ \hline
        \end{tabular}
        \end{center}

    
        \item Data ketiga: \\
        \begin{itemize}
            \item Metode Trapezoida \\
            Dengan $\Delta x = 1,$
            \begin{align*}
                I &= \int_{1}^{7} f(x) dx \\
                &\approx \frac{\Delta x}{2} \left [f(1) + f(7) + 2 \left [f(2) + f(3) + f(4) + f(5) + f(6)\right ] \right ] \\
                &\approx \frac{1}{2} \left [\num{1,8419} + \num{54,8933} + 2 \left [\num{5,6838} + \num{11,5257} + \num{19,3676} + \num{29,2095} + \num{41,0514}\right ] \right ] \\
                &\approx \frac{1}{2} \left [\num{1,8419} + \num{54,8933} + 2 \left [\num{106,8379}\right ] \right ] \\
                &\approx \frac{1}{2} (\num{269,4109}) \\
                &\approx \num{134,7054}
            \end{align*}

            \item Metode Simpson 1/3 \\
            Dengan $\Delta x = 1,$ (jumlah segmen genap: 6)
            \begin{align*}
                I &= \int_{1}^{7} f(x) dx \\
                &\approx \frac{\Delta x}{3} [f(1) + 4[f(2) + f(4) + f(6)] + 2[f(3) + f(5)] + f(7)] \\
                &\approx \frac{1}{3} [\num{1,8419} + 4[\num{5,6838} + \num{19,3676} + \num{41,0514}] + 2[\num{11,5257} + \num{29,2095}] + \num{54,8933}] \\
                &\approx \frac{1}{3} [\num{1,8419} + 4[\num{66,1028}] + 2[\num{40,7352}] + \num{54,8933}] \\ 
                &\approx \frac{1}{3} (\num{401,6396}) \\
                &\approx \num{133,8799}
            \end{align*}

            \item Metode Simpson 3/8 \\
            Dengan $\Delta x = 1,$ (jumlah segmen kelipatan 3: 6)
            \begin{align*}
                I &= \int_{1}^{7} f(x) dx \\
                &\approx \frac{3\Delta x}{8} [f(1) + 3[f(2) + f(3) + f(5) + f(6)] + 2[f(4)] + f(7)] \\
                &\approx \frac{3}{8} [\num{1,8419} + 3(\num{5,6838} + \num{11,5257} + \num{29,2095} + \num{41,0514}) + 2(\num{19,3676}) + \num{54,8933}] \\
                &\approx \frac{3}{8} [\num{1,8419} + 3(\num{87,4704}) + \num{38,7352} + \num{54,8933}] \\
                &\approx \frac{3}{8} [\num{1,8419} + \num{262,4112} + \num{38,7352} + \num{54,8933}] \\
                &\approx \frac{3}{8} (\num{357,8816}) \\
                &\approx \num{134,2056}
            \end{align*}
        \end{itemize}

    \end{itemize}


\end{enumerate}

\end{document}