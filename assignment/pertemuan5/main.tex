\documentclass{article}
\usepackage[utf8]{inputenc}
\usepackage[a4paper, margin=1in]{geometry}
\usepackage{siunitx}
\usepackage{amsmath}
\usepackage{enumitem}
\usepackage{esdiff}
\usepackage{pgfplots}
\usepackage{listings}
\usepackage{xcolor}

\pgfplotsset{compat=1.18, width=10cm}

\tolerance=1
\emergencystretch=\maxdimen
\hyphenpenalty=10000
\hbadness=10000

\sisetup{
    input-ignore={.},
    output-decimal-marker={,},
    group-minimum-digits=4,
    group-separator={.},
    group-digits=integer
}

\definecolor{darkgray}{rgb}{.4,.4,.4}

\lstdefinestyle{code}{
    aboveskip={1.3\baselineskip},
    basicstyle=\normalsize\ttfamily\linespread{4},
    breaklines=false,
    columns=fullflexible,
    commentstyle=\color[rgb]{0.127,0.427,0.514}\ttfamily\itshape,
    escapechar=@,
    extendedchars=true,
    frame=single,
    identifierstyle=\color{black},
    inputencoding=latin1,
    keywordstyle=\color[HTML]{228B22}\bfseries,
    ndkeywordstyle=\color[HTML]{228B22}\bfseries,
    numbers=left,
    numberstyle=\normalsize,
    prebreak = \raisebox{0ex}[0ex][0ex]{\ensuremath{\hookleftarrow}},
    stringstyle=\color[rgb]{0.639,0.082,0.082}\ttfamily,
    upquote=true,
    showstringspaces=false,
    xleftmargin=5ex,
    aboveskip=5pt
}

\newcommand{\penyelesaian}{\textbf{Penyelesaian: }}

\title{\textbf{Komputasi Numerik: Tugas 5}}
\author{Kelompok 15}
\date{}

\begin{document}

\maketitle

\begin{enumerate}
    \item Carilah $\int f(x) dx$ dari data-data berikut dengan batas $x=1$ sampai $x=7$ menggunakan integrasi Trapezoida, Simpson 1/3, dan Simpson 3/8 jika diketahui data-data berikut: \\
    \begin{tabular}{ c c c c c c c c }
        $x$ & $\num{1,0}$ & $\num{2,0}$ & $\num{3,0}$ & $\num{4,0}$ & $\num{5,0}$ & $\num{6,0}$ & $\num{7,0}$\\
        $f(x)$ & \num{1,8287} & \num{5,6575} & \num{11,4862} & \num{19,3149} & \num{29,1437} & \num{40,9724} & \num{54,8011} \\
    \end{tabular}
    \begin{tabular}{ c c c c c c c c }
        $x$ & $\num{1,0}$ & $\num{2,0}$ & $\num{3,0}$ & $\num{4,0}$ & $\num{5,0}$ & $\num{6,0}$ & $\num{7,0}$\\
        $f(x)$ & \num{2,1353} & \num{6,2707} & \num{12,4060} & \num{20,5413} & \num{30,6767} & \num{42,8120} & \num{56,9473} \\
    \end{tabular}
    \begin{tabular}{ c c c c c c c c }
        $x$ & $\num{1,0}$ & $\num{2,0}$ & $\num{3,0}$ & $\num{4,0}$ & $\num{5,0}$ & $\num{6,0}$ & $\num{7,0}$\\
        $f(x)$ & \num{1,8419} & \num{5,6838} & \num{11,5257} & \num{19,3676} & \num{29,2095} & \num{41,0514} & \num{54,8933} \\
    \end{tabular} \\
    \penyelesaian 

\end{enumerate}

\end{document}