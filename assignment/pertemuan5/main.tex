\documentclass{article}
\usepackage[utf8]{inputenc}
\usepackage[a4paper, margin=1in]{geometry}
\usepackage{siunitx}
\usepackage{amsmath}
\usepackage{enumitem}
\usepackage{esdiff}
\usepackage{pgfplots}
\usepackage{listings}
\usepackage{xcolor}

\pgfplotsset{compat=1.18, width=10cm}

\tolerance=1
\emergencystretch=\maxdimen
\hyphenpenalty=10000
\hbadness=10000

\sisetup{
    input-ignore={.},
    output-decimal-marker={,},
    group-minimum-digits=4,
    group-separator={.},
    group-digits=integer
}

\definecolor{darkgray}{rgb}{.4,.4,.4}

\lstdefinestyle{code}{
    aboveskip={1.3\baselineskip},
    basicstyle=\normalsize\ttfamily\linespread{4},
    breaklines=false,
    columns=fullflexible,
    commentstyle=\color[rgb]{0.127,0.427,0.514}\ttfamily\itshape,
    escapechar=@,
    extendedchars=true,
    frame=single,
    identifierstyle=\color{black},
    inputencoding=latin1,
    keywordstyle=\color[HTML]{228B22}\bfseries,
    ndkeywordstyle=\color[HTML]{228B22}\bfseries,
    numbers=left,
    numberstyle=\normalsize,
    prebreak = \raisebox{0ex}[0ex][0ex]{\ensuremath{\hookleftarrow}},
    stringstyle=\color[rgb]{0.639,0.082,0.082}\ttfamily,
    upquote=true,
    showstringspaces=false,
    xleftmargin=5ex,
    aboveskip=5pt
}

\newcommand{\penyelesaian}{\textbf{Penyelesaian: }}

\title{\textbf{Komputasi Numerik: Tugas 5}}
\author{Kelompok 15}
\date{}

\begin{document}

\maketitle

\begin{enumerate}
    \item Carilah $\int f(x) dx$ dari data-data berikut dengan batas $x=1$ sampai $x=7$ menggunakan integrasi Trapezoida, Simpson 1/3, dan Simpson 3/8 jika diketahui data-data berikut: \\
    \begin{tabular}{ c c c c c c c c }
        $x$ & $\num{1,0}$ & $\num{2,0}$ & $\num{3,0}$ & $\num{4,0}$ & $\num{5,0}$ & $\num{6,0}$ & $\num{7,0}$\\
        $f(x)$ & \num{1,8287} & \num{5,6575} & \num{11,4862} & \num{19,3149} & \num{29,1437} & \num{40,9724} & \num{54,8011} \\
    \end{tabular}
    \begin{tabular}{ c c c c c c c c }
        $x$ & $\num{1,0}$ & $\num{2,0}$ & $\num{3,0}$ & $\num{4,0}$ & $\num{5,0}$ & $\num{6,0}$ & $\num{7,0}$\\
        $f(x)$ & \num{2,1353} & \num{6,2707} & \num{12,4060} & \num{20,5413} & \num{30,6767} & \num{42,8120} & \num{56,9473} \\
    \end{tabular}
    \begin{tabular}{ c c c c c c c c }
        $x$ & $\num{1,0}$ & $\num{2,0}$ & $\num{3,0}$ & $\num{4,0}$ & $\num{5,0}$ & $\num{6,0}$ & $\num{7,0}$\\
        $f(x)$ & \num{1,8419} & \num{5,6838} & \num{11,5257} & \num{19,3676} & \num{29,2095} & \num{41,0514} & \num{54,8933} \\
    \end{tabular} \\
    \penyelesaian
    \begin{itemize}
        \item Data pertama: \\
        \begin{itemize}
            \item Metode Trapezoida \\
            Dengan $\Delta x = 1,$
            \begin{align*}
                I &= \int_{1}^{7} f(x) dx \\
                &\approx \frac{\Delta x}{2} \left [f(1) + f(7) + 2 \left [f(2) + f(3) + f(4) + f(5) + f(6)\right ] \right ] \\
                &\approx \frac{1}{2} \left [\num{1,8287} + \num{54,8011} + 2 \left [\num{5,6575} + \num{11,4862} + \num{19,3149} + \num{29,1437} + \num{40,9724}\right ] \right ] \\
                &\approx \frac{1}{2} \left [\num{1,8287} + \num{54,8011} + 2 \left [\num{106,5747}\right ] \right ] \\
                &\approx \frac{1}{2} (\num{269,7792}) \\
                &\approx \num{134,8896}
            \end{align*}
            \item Metode Simpson 1/3 \\
            Dengan $\Delta x = 1,$
            \begin{align*}
                I &= \int_{1}^{7} f(x) dx \\
                &\approx \frac{\Delta x}{3} [f(1) + 4[f(2) + f(4) + f(6)] + 2[f(3) + f(5)] + f(7)] \\
                &\approx \frac{1}{3} [\num{1,8287} + 4[\num{5,6575} + \num{19,3149} + \num{40,9724}] + 2[\num{11,4862} + \num{29,1437}] + \num{54,8011}] \\
                &\approx \frac{1}{3} [\num{1,8287} + 4[\num{65,9448}] + 2[\num{40,6299}] + \num{54,8011}] \\ 
                &\approx \frac{1}{3} (\num{401,6688}) \\
                &\approx \num{133,8896}
            \end{align*}
            \item Metode Simpson 3/8 \\
            Dengan $\Delta x = 1,$
            \begin{align*}
                I &= \int_{1}^{7} f(x) dx \\
                &\approx \frac{3\Delta x}{8} [f(1) + 3[f(2) + f(3) + f(5) + f(6)] + 3[f(4)] + f(7)] \\
                &\approx \frac{3}{8} [\num{1,8287} + 3(\num{5,6575} + \num{11,4862} + \num{29,1437} + \num{40,9724}) + 2(\num{19,3149}) + \num{54,8011}] \\
                &\approx \frac{3}{8} [\num{1,8287} + 3(\num{87,2598}) + \num{38,6298} + \num{54,8011}] \\
                &\approx \frac{3}{8} [\num{1,8287} + \num{261,7794} + \num{38,6298} + \num{54,8011}] \\
                &\approx \frac{3}{8} (\num{357,0390}) \\
                &\approx \num{133,8896}
            \end{align*}

        \end{itemize}
        \item Data kedua: \\
        \begin{itemize}
            \item Metode Trapezoida \\
            \item Metode Simpson 1/3 \\
            \item Metode Simpson 3/8 \\
        \end{itemize}
        \item Data ketiga: \\
        \begin{itemize}
            \item Metode Trapezoida \\
            \item Metode Simpson 1/3 \\
            \item Metode Simpson 3/8 \\
        \end{itemize}

    \end{itemize}


\end{enumerate}

\end{document}