\documentclass{article}
\usepackage[utf8]{inputenc}
\usepackage[a4paper, margin=1in]{geometry}
\usepackage{siunitx}
\usepackage{amsmath}
\usepackage{enumitem}
\usepackage{esdiff}

\tolerance=1
\emergencystretch=\maxdimen
\hyphenpenalty=10000
\hbadness=10000

\sisetup{
    input-ignore={.},
    output-decimal-marker={,},
    group-minimum-digits=4,
    group-separator={.},
    group-digits=integer
}

\newcommand{\penyelesaian}{\textbf{Penyelesaian: }}

\title{\textbf{Komputasi Numerik: Tugas x}}
\author{Kelompok 15}
\date{}

\begin{document}

\maketitle

\begin{enumerate}
    \item Soal 1
    \begin{enumerate}
        \item Subsoal (a) \\
        \penyelesaian Penyelesaian soal 1 subsoal (a)

        \item Subsoal (b) \\
        \penyelesaian Penyelesaian soal 1 subsoal (b)

        \item Subsoal (c) \\
        \penyelesaian  Penyelesaian soal 1 subsoal (c)
    \end{enumerate}

    \item Sekarang lengkapi jawaban no. 1 di atas dengan metode Tabulasi. \\
    \penyelesaian 
    \begin{enumerate}
        \item $e^x - x - 2 = 0$ \\
        Tabulasi akar $x_1$: \\
        \begin{tabular}{|c|c|}
            \hline
            $x$   & $f(x)$ \\
            \hline
            \num{-2,0} & \num{0,1353} \\
            \textcolor{red}{\num{-1,9}} & \textcolor{red}{\num{0,0496}} \\
            \textcolor{red}{\num{-1,8}} & \textcolor{red}{\num{-0,0347}} \\
            \num{-1,7} & \num{-0,1173} \\
            \num{-1,6} & \num{-0,1981} \\
            \num{-1,5} & \num{-0,2769} \\
            \num{-1,4} & \num{-0,3534} \\
            \num{-1,3} & \num{-0,4275} \\
            \num{-1,2} & \num{-0,4988} \\
            \num{-1,1} & \num{-0,5671} \\
            \num{-1,0} & \num{-0,6321} \\
            \hline
            \end{tabular}\quad
            \begin{tabular}{|c|c|}
            \hline
            $x$   & $f(x)$ \\
            \hline
            \num{-1,90} & \num{0,0496} \\
            \num{-1,89} & \num{0,0411} \\
            \num{-1,88} & \num{0,0326} \\
            \num{-1,87} & \num{0,0241} \\
            \num{-1,86} & \num{0,0157} \\
            \textcolor{red}{\num{-1,85}} & \textcolor{red}{\num{0,0072}} \\
            \textcolor{red}{\num{-1,84}} & \textcolor{red}{\num{-0,0012}} \\
            \num{-1,83} & \num{-0,0096} \\
            \num{-1,82} & \num{-0,0180} \\
            \num{-1,81} & \num{-0,0263} \\
            \num{-1,80} & \num{-0,0347} \\
            \hline
            \end{tabular}\quad
            \begin{tabular}{|c|c|}
            \hline
            $x$   & $f(x)$ \\
            \hline
            \num{-1,850} & \num{0,0072} \\
            \num{-1,849} & \num{0,0064} \\
            \num{-1,848} & \num{0,0056} \\
            \num{-1,847} & \num{0,0047} \\
            \num{-1,846} & \num{0,0039} \\
            \num{-1,845} & \num{0,0030} \\
            \num{-1,844} & \num{0,0022} \\
            \num{-1,843} & \num{0,0013} \\
            \textcolor{red}{\num{-1,842}} & \textcolor{red}{\num{0,0005}} \\
            \textcolor{red}{\num{-1,841}} & \textcolor{red}{\num{-0,0003}} \\
            \num{-1,840} & \num{-0,0012} \\
            \hline
            \end{tabular}\quad
            \begin{tabular}{|c|c|}
            \hline
            $x$   & $f(x)$ \\
            \hline
            \num{-1,8420} & \num{0,0005} \\
            \num{-1,8419} & \num{0,0004} \\
            \num{-1,8418} & \num{0,0003} \\
            \num{-1,8417} & \num{0,0002} \\
            \num{-1,8416} & \num{0,0002} \\
            \textcolor{red}{\num{-1,8415}} & \textcolor{red}{\num{0,0001}} \\
            \textcolor{red}{\num{-1,8414}} & \textcolor{red}{\num{-0,0000}} \\
            \num{-1,8413} & \num{-0,0001} \\
            \num{-1,8412} & \num{-0,0002} \\
            \num{-1,8411} & \num{-0,0003} \\
            \num{-1,8410} & \num{-0,0003} \\
            \hline
            \end{tabular}\quad \\
    
        Tabulasi akar $x_2$: \\ 
        \begin{tabular}{|c|c|}
            \hline
            $x$   & $f(x)$ \\
            \hline
            \num{1,0} & \num{-0,2817} \\
            \textcolor{red}{\num{1,1}} & \textcolor{red}{\num{-0,0958}} \\
            \textcolor{red}{\num{1,2}} & \textcolor{red}{\num{0,1201}} \\
            \num{1,3} & \num{0,3693} \\
            \num{1,4} & \num{0,6552} \\
            \num{1,5} & \num{0,9817} \\
            \num{1,6} & \num{1,3530} \\
            \num{1,7} & \num{1,7739} \\
            \num{1,8} & \num{2,2496} \\
            \num{1,9} & \num{2,7859} \\
            \num{2,0} & \num{3,3891} \\
            \hline
            \end{tabular}\quad
            \begin{tabular}{|c|c|}
            \hline
            $x$   & $f(x)$ \\
            \hline
            \num{1,10} & \num{-0,0958} \\
            \num{1,11} & \num{-0,0756} \\
            \num{1,12} & \num{-0,0551} \\
            \num{1,13} & \num{-0,0343} \\
            \textcolor{red}{\num{1,14}} & \textcolor{red}{\num{-0,0132}} \\
            \textcolor{red}{\num{1,15}} & \textcolor{red}{\num{0,0082}} \\
            \num{1,16} & \num{0,0299} \\
            \num{1,17} & \num{0,0520} \\
            \num{1,18} & \num{0,0744} \\
            \num{1,19} & \num{0,0971} \\
            \num{1,20} & \num{0,1201} \\
            \hline
            \end{tabular}\quad
            \begin{tabular}{|c|c|}
            \hline
            $x$   & $f(x)$ \\
            \hline
            \num{1,140} & \num{-0,0132} \\
            \num{1,141} & \num{-0,0111} \\
            \num{1,142} & \num{-0,0090} \\
            \num{1,143} & \num{-0,0068} \\
            \num{1,144} & \num{-0,0047} \\
            \num{1,145} & \num{-0,0026} \\
            \textcolor{red}{\num{1,146}} & \textcolor{red}{\num{-0,0004}} \\
            \textcolor{red}{\num{1,147}} & \textcolor{red}{\num{0,0017}} \\
            \num{1,148} & \num{0,0039} \\
            \num{1,149} & \num{0,0060} \\
            \num{1,150} & \num{0,0082} \\
            \hline
            \end{tabular}\quad
            \begin{tabular}{|c|c|}
            \hline
            $x$   & $f(x)$ \\
            \hline
            \num{1,1460} & \num{-0,0004} \\
            \textcolor{red}{\num{1,1461}} & \textcolor{red}{\num{-0,0002}} \\
            \textcolor{red}{\num{1,1462}} & \textcolor{red}{\num{0,0000}} \\
            \num{1,1463} & \num{0,0002} \\
            \num{1,1464} & \num{0,0004} \\
            \num{1,1465} & \num{0,0007} \\
            \num{1,1466} & \num{0,0009} \\
            \num{1,1467} & \num{0,0011} \\
            \num{1,1468} & \num{0,0013} \\
            \num{1,1469} & \num{0,0015} \\
            \num{1,1470} & \num{0,0017} \\
            \hline
            \end{tabular}\quad \\
    
        Dengan demikian, akar-akar dari $e^x - x - 2 = 0$, antara lain $x_1 \approx \num{-1,8414}$ dan $x_2 \approx \num{1,1462}$.\\
    
        \item $10^x = 100 - 2x$. \\
        Tabulasi akar $x$: \\
        \begin{tabular}{|c|c|}
            \hline
            $x$   & $f(x)$ \\
            \hline
            \num{1,90} & \num{-16,7672} \\
            \num{1,91} & \num{-14,8969} \\
            \num{1,92} & \num{-12,9836} \\
            \num{1,93} & \num{-11,0262} \\
            \num{1,94} & \num{-9,0236} \\
            \num{1,95} & \num{-6,9749} \\
            \num{1,96} & \num{-4,8789} \\
            \num{1,97} & \num{-2,7346} \\
            \textcolor{red}{\num{1,98}} & \textcolor{red}{\num{-0,5407}} \\
            \textcolor{red}{\num{1,99}} & \textcolor{red}{\num{1,7037}} \\
            \num{2,00} & \num{4,0000} \\
            \hline
            \end{tabular}\quad
            \begin{tabular}{|c|c|}
            \hline
            $x$   & $f(x)$ \\
            \hline
            \num{1,980} & \num{-0,5407} \\
            \num{1,981} & \num{-0,3186} \\
            \textcolor{red}{\num{1,982}} & \textcolor{red}{\num{-0,0959}} \\
            \textcolor{red}{\num{1,983}} & \textcolor{red}{\num{0,1272}} \\
            \num{1,984} & \num{0,3509} \\
            \num{1,985} & \num{0,5751} \\
            \num{1,986} & \num{0,7998} \\
            \num{1,987} & \num{1,0250} \\
            \num{1,988} & \num{1,2507} \\
            \num{1,989} & \num{1,4770} \\
            \num{1,990} & \num{1,7037} \\
            \hline
            \end{tabular}\quad
            \begin{tabular}{|c|c|}
            \hline
            $x$   & $f(x)$ \\
            \hline
            \num{1,9820} & \num{-0,0959} \\
            \num{1,9821} & \num{-0,0736} \\
            \num{1,9822} & \num{-0,0513} \\
            \num{1,9823} & \num{-0,0290} \\
            \textcolor{red}{\num{1,9824}} & \textcolor{red}{\num{-0,0067}} \\
            \textcolor{red}{\num{1,9825}} & \textcolor{red}{\num{0,0156}} \\
            \num{1,9826} & \num{0,0379} \\
            \num{1,9827} & \num{0,0602} \\
            \num{1,9828} & \num{0,0826} \\
            \num{1,9829} & \num{0,1049} \\
            \num{1,9830} & \num{0,1272} \\
            \hline
            \end{tabular}\quad
            \begin{tabular}{|c|c|}
            \hline
            $x$   & $f(x)$ \\
            \hline
            \num{1,98240} & \num{-0,0067} \\
            \num{1,98241} & \num{-0,0045} \\
            \num{1,98242} & \num{-0,0023} \\
            \textcolor{red}{\num{1,98243}} & \textcolor{red}{\num{-0,0000}} \\
            \textcolor{red}{\num{1,98244}} & \textcolor{red}{\num{0,0022}} \\
            \num{1,98245} & \num{0,0044} \\
            \num{1,98246} & \num{0,0067} \\
            \num{1,98247} & \num{0,0089} \\
            \num{1,98248} & \num{0,0111} \\
            \num{1,98249} & \num{0,0134} \\
            \num{1,98250} & \num{0,0156} \\
            \hline
            \end{tabular}\quad \\
            
        Dengan demikian, akar dari $10^x = 100 - 2x$ adalah $x \approx \num{1,98243}$.
    
        \item $\num{-0,874}x^2 + \num{1,75}x + \num{2,627}$ \\
        Tabulasi akar $x_1$:\\
        \begin{tabular}{|c|c|}
            \hline
            $x$   & $f(x)$ \\
            \hline
            \num{-1,10} & \num{-0,3555} \\
            \num{-1,09} & \num{-0,3189} \\
            \num{-1,08} & \num{-0,2824} \\
            \num{-1,07} & \num{-0,2461} \\
            \num{-1,06} & \num{-0,2100} \\
            \num{-1,05} & \num{-0,1741} \\
            \num{-1,04} & \num{-0,1383} \\
            \num{-1,03} & \num{-0,1027} \\
            \num{-1,02} & \num{-0,0673} \\
            \textcolor{red}{\num{-1,01}} & \textcolor{red}{\num{-0,0321}} \\
            \textcolor{red}{\num{-1,00}} & \textcolor{red}{\num{0,0030}} \\
            \hline
            \end{tabular}\quad
            \begin{tabular}{|c|c|}
            \hline
            $x$   & $f(x)$ \\
            \hline
            \num{-1,010} & \num{-0,0321} \\
            \num{-1,009} & \num{-0,0286} \\
            \num{-1,008} & \num{-0,0250} \\
            \num{-1,007} & \num{-0,0215} \\
            \num{-1,006} & \num{-0,0180} \\
            \num{-1,005} & \num{-0,0145} \\
            \num{-1,004} & \num{-0,0110} \\
            \num{-1,003} & \num{-0,0075} \\
            \num{-1,002} & \num{-0,0040} \\
            \textcolor{red}{\num{-1,001}} & \textcolor{red}{\num{-0,0005}} \\
            \textcolor{red}{\num{-1,000}} & \textcolor{red}{\num{0,0030}} \\
            \hline
            \end{tabular}\quad
            \begin{tabular}{|c|c|}
            \hline
            $x$   & $f(x)$ \\
            \hline
            \num{-1,0010} & \num{-0,0005} \\
            \textcolor{red}{\num{-1,0009}} & \textcolor{red}{\num{-0,0001}} \\
            \textcolor{red}{\num{-1,0008}} & \textcolor{red}{\num{0,0002}} \\
            \num{-1,0007} & \num{0,0006} \\
            \num{-1,0006} & \num{0,0009} \\
            \num{-1,0005} & \num{0,0013} \\
            \num{-1,0004} & \num{0,0016} \\
            \num{-1,0003} & \num{0,0020} \\
            \num{-1,0002} & \num{0,0023} \\
            \num{-1,0001} & \num{0,0027} \\
            \num{-1,0000} & \num{0,0030} \\
            \hline
            \end{tabular}\quad
            \begin{tabular}{|c|c|}
            \hline
            $x$   & $f(x)$ \\
            \hline
            \num{-1,00090} & \num{-0,0001} \\
            \num{-1,00089} & \num{-0,0001} \\
            \num{-1,00088} & \num{-0,0001} \\
            \num{-1,00087} & \num{-0,0000} \\
            \textcolor{red}{\num{-1,00086}} & \textcolor{red}{\num{-0,0000}} \\
            \textcolor{red}{\num{-1,00085}} & \textcolor{red}{\num{0,0000}} \\
            \num{-1,00084} & \num{0,0001} \\
            \num{-1,00083} & \num{0,0001} \\
            \num{-1,00082} & \num{0,0001} \\
            \num{-1,00081} & \num{0,0002} \\
            \num{-1,00080} & \num{0,0002} \\
            \hline
            \end{tabular}\quad \\            
            
            Tabulasi akar $x_2$: \\
            \begin{tabular}{|c|c|}
                \hline
                $x$   & $f(x)$ \\
                \hline
                \textcolor{red}{\num{3,00}} & \textcolor{red}{\num{0,0110}} \\
                \textcolor{red}{\num{3,01}} & \textcolor{red}{\num{-0,0240}} \\
                \num{3,02} & \num{-0,0592} \\
                \num{3,03} & \num{-0,0946} \\
                \num{3,04} & \num{-0,1302} \\
                \num{3,05} & \num{-0,1659} \\
                \num{3,06} & \num{-0,2018} \\
                \num{3,07} & \num{-0,2379} \\
                \num{3,08} & \num{-0,2741} \\
                \num{3,09} & \num{-0,3105} \\
                \num{3,10} & \num{-0,3471} \\
                \hline
                \end{tabular}\quad
                \begin{tabular}{|c|c|}
                \hline
                $x$   & $f(x)$ \\
                \hline
                \num{3,000} & \num{0,0110} \\
                \num{3,001} & \num{0,0075} \\
                \num{3,002} & \num{0,0040} \\
                \textcolor{red}{\num{3,003}} & \textcolor{red}{\num{0,0005}} \\
                \textcolor{red}{\num{3,004}} & \textcolor{red}{\num{-0,0030}} \\
                \num{3,005} & \num{-0,0065} \\
                \num{3,006} & \num{-0,0100} \\
                \num{3,007} & \num{-0,0135} \\
                \num{3,008} & \num{-0,0170} \\
                \num{3,009} & \num{-0,0205} \\
                \num{3,010} & \num{-0,0240} \\
                \hline
                \end{tabular}\quad
                \begin{tabular}{|c|c|}
                \hline
                $x$   & $f(x)$ \\
                \hline
                \num{3,0030} & \num{0,0005} \\
                \textcolor{red}{\num{3,0031}} & \textcolor{red}{\num{0,0002}} \\
                \textcolor{red}{\num{3,0032}} & \textcolor{red}{\num{-0,0002}} \\
                \num{3,0033} & \num{-0,0005} \\
                \num{3,0034} & \num{-0,0009} \\
                \num{3,0035} & \num{-0,0012} \\
                \num{3,0036} & \num{-0,0016} \\
                \num{3,0037} & \num{-0,0019} \\
                \num{3,0038} & \num{-0,0023} \\
                \num{3,0039} & \num{-0,0026} \\
                \num{3,0040} & \num{-0,0030} \\
                \hline
                \end{tabular}\quad
                \begin{tabular}{|c|c|}
                \hline
                $x$   & $f(x)$ \\
                \hline
                \num{3,00310} & \num{0,0002} \\
                \num{3,00311} & \num{0,0001} \\
                \num{3,00312} & \num{0,0001} \\
                \num{3,00313} & \num{0,0001} \\
                \textcolor{red}{\num{3,00314}} & \textcolor{red}{\num{0,0000}} \\
                \textcolor{red}{\num{3,00315}} & \textcolor{red}{\num{-0,0000}} \\
                \num{3,00316} & \num{-0,0000} \\
                \num{3,00317} & \num{-0,0001} \\
                \num{3,00318} & \num{-0,0001} \\
                \num{3,00319} & \num{-0,0002} \\
                \num{3,00320} & \num{-0,0002} \\
                \hline
                \end{tabular}\quad \\                   
    
            Dengan demikian, akar-akarnya adalah $x_1 \approx \num{-1,00086}$ dan $x_2 \approx \num{3,00314}$.
    
        \item $\num{-2,1} + \num{6,21}x - \num{3,9}x^2 + \num{0,667}x^3$ \\
        Tabulasi akar $x_1$: \\
        \begin{tabular}{|c|c|}
            \hline
            $x$   & $f(x)$ \\
            \hline
            \num{0,40} & \num{-0,1973} \\
            \num{0,41} & \num{-0,1635} \\
            \num{0,42} & \num{-0,1303} \\
            \num{0,43} & \num{-0,0978} \\
            \num{0,44} & \num{-0,0658} \\
            \num{0,45} & \num{-0,0345} \\
            \textcolor{red}{\num{0,46}} & \textcolor{red}{\num{-0,0037}} \\
            \textcolor{red}{\num{0,47}} & \textcolor{red}{\num{0,0264}} \\
            \num{0,48} & \num{0,0560} \\
            \num{0,49} & \num{0,0850} \\
            \num{0,50} & \num{0,1134} \\
            \hline
            \end{tabular}\quad
            \begin{tabular}{|c|c|}
            \hline
            $x$   & $f(x)$ \\
            \hline
            \num{0,460} & \num{-0,0037} \\
            \textcolor{red}{\num{0,461}} & \textcolor{red}{\num{-0,0007}} \\
            \textcolor{red}{\num{0,462}} & \textcolor{red}{\num{0,0024}} \\
            \num{0,463} & \num{0,0054} \\
            \num{0,464} & \num{0,0084} \\
            \num{0,465} & \num{0,0114} \\
            \num{0,466} & \num{0,0144} \\
            \num{0,467} & \num{0,0175} \\
            \num{0,468} & \num{0,0205} \\
            \num{0,469} & \num{0,0235} \\
            \num{0,470} & \num{0,0264} \\
            \hline
            \end{tabular}\quad
            \begin{tabular}{|c|c|}
            \hline
            $x$   & $f(x)$ \\
            \hline
            \num{0,4610} & \num{-0,0007} \\
            \num{0,4611} & \num{-0,0004} \\
            \textcolor{red}{\num{0,4612}} & \textcolor{red}{\num{-0,0001}} \\
            \textcolor{red}{\num{0,4613}} & \textcolor{red}{\num{0,0002}} \\
            \num{0,4614} & \num{0,0005} \\
            \num{0,4615} & \num{0,0008} \\
            \num{0,4616} & \num{0,0011} \\
            \num{0,4617} & \num{0,0015} \\
            \num{0,4618} & \num{0,0018} \\
            \num{0,4619} & \num{0,0021} \\
            \num{0,4620} & \num{0,0024} \\
            \hline
            \end{tabular}\quad
            \begin{tabular}{|c|c|}
            \hline
            $x$   & $f(x)$ \\
            \hline
            \num{0,46120} & \num{-0,0001} \\
            \num{0,46121} & \num{-0,0000} \\
            \textcolor{red}{\num{0,46122}} & \textcolor{red}{\num{-0,0000}} \\
            \textcolor{red}{\num{0,46123}} & \textcolor{red}{\num{0,0000}} \\
            \num{0,46124} & \num{0,0001} \\
            \num{0,46125} & \num{0,0001} \\
            \num{0,46126} & \num{0,0001} \\
            \num{0,46127} & \num{0,0001} \\
            \num{0,46128} & \num{0,0002} \\
            \num{0,46129} & \num{0,0002} \\
            \num{0,46130} & \num{0,0002} \\
            \hline
            \end{tabular}\quad \\
    
        Tabulasi akar $x_2$: \\
        \begin{tabular}{|c|c|}
            \hline
            $x$   & $f(x)$ \\
            \hline
            \num{2,00} & \num{0,0560} \\
            \num{2,01} & \num{0,0422} \\
            \num{2,02} & \num{0,0283} \\
            \num{2,03} & \num{0,0145} \\
            \textcolor{red}{\num{2,04}} & \textcolor{red}{\num{0,0008}} \\
            \textcolor{red}{\num{2,05}} & \textcolor{red}{\num{-0,0130}} \\
            \num{2,06} & \num{-0,0266} \\
            \num{2,07} & \num{-0,0403} \\
            \num{2,08} & \num{-0,0539} \\
            \num{2,09} & \num{-0,0674} \\
            \num{2,10} & \num{-0,0809} \\
            \hline
            \end{tabular}\quad
            \begin{tabular}{|c|c|}
            \hline
            $x$   & $f(x)$ \\
            \hline
            \textcolor{red}{\num{2,040}} & \textcolor{red}{\num{0,0008}} \\
            \textcolor{red}{\num{2,041}} & \textcolor{red}{\num{-0,0006}} \\
            \num{2,042} & \num{-0,0020} \\
            \num{2,043} & \num{-0,0034} \\
            \num{2,044} & \num{-0,0047} \\
            \num{2,045} & \num{-0,0061} \\
            \num{2,046} & \num{-0,0075} \\
            \num{2,047} & \num{-0,0088} \\
            \num{2,048} & \num{-0,0102} \\
            \num{2,049} & \num{-0,0116} \\
            \num{2,050} & \num{-0,0130} \\
            \hline
            \end{tabular}\quad
            \begin{tabular}{|c|c|}
            \hline
            $x$   & $f(x)$ \\
            \hline
            \num{2,0400} & \num{0,0008} \\
            \num{2,0401} & \num{0,0006} \\
            \num{2,0402} & \num{0,0005} \\
            \num{2,0403} & \num{0,0004} \\
            \num{2,0404} & \num{0,0002} \\
            \textcolor{red}{\num{2,0405}} & \textcolor{red}{\num{0,0001}} \\
            \textcolor{red}{\num{2,0406}} & \textcolor{red}{\num{-0,0001}} \\
            \num{2,0407} & \num{-0,0002} \\
            \num{2,0408} & \num{-0,0003} \\
            \num{2,0409} & \num{-0,0005} \\
            \num{2,0410} & \num{-0,0006} \\
            \hline
            \end{tabular}\quad
            \begin{tabular}{|c|c|}
            \hline
            $x$   & $f(x)$ \\
            \hline
            \num{2,04050} & \num{0,0001} \\
            \num{2,04051} & \num{0,0001} \\
            \num{2,04052} & \num{0,0001} \\
            \num{2,04053} & \num{0,0000} \\
            \num{2,04054} & \num{0,0000} \\
            \textcolor{red}{\num{2,04055}} & \textcolor{red}{\num{0,0000}} \\
            \textcolor{red}{\num{2,04056}} & \textcolor{red}{\num{-0,0000}} \\
            \num{2,04057} & \num{-0,0000} \\
            \num{2,04058} & \num{-0,0000} \\
            \num{2,04059} & \num{-0,0000} \\
            \num{2,04060} & \num{-0,0001} \\
            \hline
            \end{tabular}\quad \\         
            
        Dengan demikian, akar-akarnya adalah $x_1 \approx \num{0,46122}$ dan $x_2 \approx \num{2,04056}$.
    
        \item $(1 - \num{0,6}x)/x$ \\
        Tabulasi akar $x$: \\
        \begin{tabular}{|c|c|}
            \hline
            $x$   & $f(x)$ \\
            \hline
            \num{1,60} & \num{0,0250} \\
            \num{1,61} & \num{0,0211} \\
            \num{1,62} & \num{0,0173} \\
            \num{1,63} & \num{0,0135} \\
            \num{1,64} & \num{0,0098} \\
            \num{1,65} & \num{0,0061} \\
            \textcolor{red}{\num{1,66}} & \textcolor{red}{\num{0,0024}} \\
            \textcolor{red}{\num{1,67}} & \textcolor{red}{\num{-0,0012}} \\
            \num{1,68} & \num{-0,0048} \\
            \num{1,69} & \num{-0,0083} \\
            \num{1,70} & \num{-0,0118} \\
            \hline
            \end{tabular}\quad
            \begin{tabular}{|c|c|}
            \hline
            $x$   & $f(x)$ \\
            \hline
            \num{1,660} & \num{0,0024} \\
            \num{1,661} & \num{0,0020} \\
            \num{1,662} & \num{0,0017} \\
            \num{1,663} & \num{0,0013} \\
            \num{1,664} & \num{0,0010} \\
            \num{1,665} & \num{0,0006} \\
            \textcolor{red}{\num{1,666}} & \textcolor{red}{\num{0,0002}} \\
            \textcolor{red}{\num{1,667}} & \textcolor{red}{\num{-0,0001}} \\
            \num{1,668} & \num{-0,0005} \\
            \num{1,669} & \num{-0,0008} \\
            \num{1,670} & \num{-0,0012} \\
            \hline
            \end{tabular}\quad
            \begin{tabular}{|c|c|}
            \hline
            $x$   & $f(x)$ \\
            \hline
            \num{1,6660} & \num{0,0002} \\
            \num{1,6661} & \num{0,0002} \\
            \num{1,6662} & \num{0,0002} \\
            \num{1,6663} & \num{0,0001} \\
            \num{1,6664} & \num{0,0001} \\
            \num{1,6665} & \num{0,0001} \\
            \textcolor{red}{\num{1,6666}} & \textcolor{red}{\num{0,0000}} \\
            \textcolor{red}{\num{1,6667}} & \textcolor{red}{\num{-0,0000}} \\
            \num{1,6668} & \num{-0,0000} \\
            \num{1,6669} & \num{-0,0001} \\
            \num{1,6670} & \num{-0,0001} \\
            \hline
            \end{tabular}\quad
            \begin{tabular}{|c|c|}
            \hline
            $x$   & $f(x)$ \\
            \hline
            \num{1,66660} & \num{0,0000} \\
            \num{1,66661} & \num{0,0000} \\
            \num{1,66662} & \num{0,0000} \\
            \num{1,66663} & \num{0,0000} \\
            \num{1,66664} & \num{0,0000} \\
            \num{1,66665} & \num{0,0000} \\
            \textcolor{red}{\num{1,66666}} & \textcolor{red}{\num{0,0000}} \\
            \textcolor{red}{\num{1,66667}} & \textcolor{red}{\num{-0,0000}} \\
            \num{1,66668} & \num{-0,0000} \\
            \num{1,66669} & \num{-0,0000} \\
            \num{1,66670} & \num{-0,0000} \\
            \hline
            \end{tabular}\quad \\ 
    
        Dengan demikian, akar dari $(1 - \num{0,6}x) / x$ adalah $x \approx \num{1,66667}$.
    
    \end{enumerate}
\end{enumerate}

\end{document}