\documentclass{article}
\usepackage[utf8]{inputenc}
\usepackage[a4paper, margin=1in]{geometry}
\usepackage{siunitx}
\usepackage{amsmath}
\usepackage{enumitem}
\usepackage{esdiff}

\tolerance=1
\emergencystretch=\maxdimen
\hyphenpenalty=10000
\hbadness=10000

\sisetup{
    input-ignore={.},
    output-decimal-marker={,},
    group-minimum-digits=4,
    group-separator={.},
    group-digits=integer
}

\newcommand{\penyelesaian}{\textbf{Penyelesaian: }}

\title{\textbf{Komputasi Numerik: Tugas 2}}
\author{Kelompok 15}
\date{}

\begin{document}

\maketitle

\begin{enumerate}
    \item Soal 1
    \begin{enumerate}
        \item Subsoal (a) \\
        \penyelesaian Penyelesaian soal 1 subsoal (a)

        \item Subsoal (b) \\
        \penyelesaian Penyelesaian soal 1 subsoal (b)

        \item Subsoal (c) \\
        \penyelesaian  Penyelesaian soal 1 subsoal (c)
    \end{enumerate}

    \item Soal 2
    \penyelesaian Penyelesaian soal 2

    \item Dengan metode Bolzano, dapatkan akar-akar persamaan:
    \begin{enumerate}
        \item $x^3 - 3x + 1 = 0$ ($x_0 = \num{1,5}$; s.d. 3 digit) \\
        \penyelesaian \\
        \begin{tabular}{|c|c|c|c|c|c|c|}
            \hline
            iterasi & $x_1$ & $x_2$ & $x_3$ & $f(x_1)$ & $f(x_2)$ & $f(x_3)$ \\
            \hline
            1 & \num{1,000} & \num{2,000} & \num{1,500} & \num{-1,000} & \num{3,000} & \num{-0,125}\\
            2 & \num{1,500} & \num{2,000} & \num{1,750} & \num{-0,125} & \num{3,000} & \num{1,109}\\
            3 & \num{1,500} & \num{1,750} & \num{1,625} & \num{-0,125} & \num{1,109} & \num{0,416}\\
            4 & \num{1,500} & \num{1,625} & \num{1,562} & \num{-0,125} & \num{0,416} & \num{0,127}\\
            5 & \num{1,500} & \num{1,562} & \num{1,531} & \num{-0,125} & \num{0,127} & \num{-0,003}\\
            6 & \num{1,531} & \num{1,562} & \num{1,547} & \num{-0,003} & \num{0,127} & \num{0,061}\\
            7 & \num{1,531} & \num{1,547} & \num{1,539} & \num{-0,003} & \num{0,061} & \num{0,028}\\
            8 & \num{1,531} & \num{1,539} & \num{1,535} & \num{-0,003} & \num{0,028} & \num{0,012}\\
            9 & \num{1,531} & \num{1,535} & \num{1,533} & \num{-0,003} & \num{0,012} & \num{0,005}\\
            10 & \num{1,531} & \num{1,533} & \num{1,532} & \num{-0,003} & \num{0,005} & \num{0,001}\\
            11 & \num{1,531} & \num{1,532} & \num{1,532} & \num{-0,003} & \num{0,001} & \num{-0,001}\\
            12 & \num{1,532} & \num{1,532} & \num{1,532} & \num{-0,001} & \num{0,001} & \num{-0,000}\\
            13 & \num{1,532} & \num{1,532} & \num{1,532} & \num{-0,000} & \num{0,001} & \num{0,000}\\
             \hline
            \end{tabular} \\
        Dengan demikian, akar dari persamaan $x^3 - 3x + 1 = 0$ hingga 3 digit dibelakang koma adalah $x = \num{1,532}$.
            
        \item $\cos{x} = 3x$ ($x_0 = \num{0,3}$; s.d. 5 digit) \\
        \penyelesaian \\
        \begin{tabular}{|c|c|c|c|c|c|c|}
            \hline
            iterasi & $x_1$ & $x_2$ & $x_3$ & $f(x_1)$ & $f(x_2)$ & $f(x_3)$ \\
            \hline
            1 & \num{0,20000} & \num{0,40000} & \num{0,30000} & \num{0,38007} & \num{-0,27894} & \num{0,05534}\\
            2 & \num{0,30000} & \num{0,40000} & \num{0,35000} & \num{0,05534} & \num{-0,27894} & \num{-0,11063}\\
            3 & \num{0,30000} & \num{0,35000} & \num{0,32500} & \num{0,05534} & \num{-0,11063} & \num{-0,02735}\\
            4 & \num{0,30000} & \num{0,32500} & \num{0,31250} & \num{0,05534} & \num{-0,02735} & \num{0,01407}\\
            5 & \num{0,31250} & \num{0,32500} & \num{0,31875} & \num{0,01407} & \num{-0,02735} & \num{-0,00662}\\
            6 & \num{0,31250} & \num{0,31875} & \num{0,31563} & \num{0,01407} & \num{-0,00662} & \num{0,00373}\\
            7 & \num{0,31563} & \num{0,31875} & \num{0,31719} & \num{0,00373} & \num{-0,00662} & \num{-0,00145}\\
            8 & \num{0,31563} & \num{0,31719} & \num{0,31641} & \num{0,00373} & \num{-0,00145} & \num{0,00114}\\
            9 & \num{0,31641} & \num{0,31719} & \num{0,31680} & \num{0,00114} & \num{-0,00145} & \num{-0,00015}\\
            10 & \num{0,31641} & \num{0,31680} & \num{0,31660} & \num{0,00114} & \num{-0,00015} & \num{0,00049}\\
            11 & \num{0,31660} & \num{0,31680} & \num{0,31670} & \num{0,00049} & \num{-0,00015} & \num{0,00017}\\
            12 & \num{0,31670} & \num{0,31680} & \num{0,31675} & \num{0,00017} & \num{-0,00015} & \num{0,00001}\\
            13 & \num{0,31675} & \num{0,31680} & \num{0,31677} & \num{0,00001} & \num{-0,00015} & \num{-0,00007}\\
            14 & \num{0,31675} & \num{0,31677} & \num{0,31676} & \num{0,00001} & \num{-0,00007} & \num{-0,00003}\\
            15 & \num{0,31675} & \num{0,31676} & \num{0,31675} & \num{0,00001} & \num{-0,00003} & \num{-0,00001}\\
            16 & \num{0,31675} & \num{0,31675} & \num{0,31675} & \num{0,00001} & \num{-0,00001} & \num{-0,00000}\\
             \hline
            \end{tabular} \\                     
        Dengan demikian, akar dari persamaan $\cos{x} = 3x$ hingga 5 digit di belakang koma adalah $x = \num{0,31675}$.

        \item $10^x = 100 - 2x$ ($x_0 = \num{2}$; s.d. 4 digit) \\
        \penyelesaian \\         
        \begin{tabular}{|c|c|c|c|c|c|c|}
            \hline
            iterasi & $x_1$ & $x_2$ & $x_3$ & $f(x_1)$ & $f(x_2)$ & $f(x_3)$ \\
            \hline
            1 & \num{1,5000} & \num{2,5000} & \num{2,0000} & \num{-65,3772} & \num{221,2278} & \num{4,0000}\\
            2 & \num{1,5000} & \num{2,0000} & \num{1,7500} & \num{-65,3772} & \num{4,0000} & \num{-40,2659}\\
            3 & \num{1,7500} & \num{2,0000} & \num{1,8750} & \num{-40,2659} & \num{4,0000} & \num{-21,2606}\\
            4 & \num{1,8750} & \num{2,0000} & \num{1,9375} & \num{-21,2606} & \num{4,0000} & \num{-9,5286}\\
            5 & \num{1,9375} & \num{2,0000} & \num{1,9688} & \num{-9,5286} & \num{4,0000} & \num{-3,0053}\\
            6 & \num{1,9688} & \num{2,0000} & \num{1,9844} & \num{-3,0053} & \num{4,0000} & \num{0,4349}\\
            7 & \num{1,9688} & \num{1,9844} & \num{1,9766} & \num{-3,0053} & \num{0,4349} & \num{-1,3005}\\
            8 & \num{1,9766} & \num{1,9844} & \num{1,9805} & \num{-1,3005} & \num{0,4349} & \num{-0,4367}\\
            9 & \num{1,9805} & \num{1,9844} & \num{1,9824} & \num{-0,4367} & \num{0,4349} & \num{-0,0019}\\
            10 & \num{1,9824} & \num{1,9844} & \num{1,9834} & \num{-0,0019} & \num{0,4349} & \num{0,2163}\\
            11 & \num{1,9824} & \num{1,9834} & \num{1,9829} & \num{-0,0019} & \num{0,2163} & \num{0,1072}\\
            12 & \num{1,9824} & \num{1,9829} & \num{1,9827} & \num{-0,0019} & \num{0,1072} & \num{0,0526}\\
            13 & \num{1,9824} & \num{1,9827} & \num{1,9825} & \num{-0,0019} & \num{0,0526} & \num{0,0254}\\
            14 & \num{1,9824} & \num{1,9825} & \num{1,9825} & \num{-0,0019} & \num{0,0254} & \num{0,0118}\\
            15 & \num{1,9824} & \num{1,9825} & \num{1,9825} & \num{-0,0019} & \num{0,0118} & \num{0,0050}\\
            16 & \num{1,9824} & \num{1,9825} & \num{1,9824} & \num{-0,0019} & \num{0,0050} & \num{0,0016}\\
            17 & \num{1,9824} & \num{1,9824} & \num{1,9824} & \num{-0,0019} & \num{0,0016} & \num{-0,0001}\\
            18 & \num{1,9824} & \num{1,9824} & \num{1,9824} & \num{-0,0001} & \num{0,0016} & \num{0,0007}\\
            19 & \num{1,9824} & \num{1,9824} & \num{1,9824} & \num{-0,0001} & \num{0,0007} & \num{0,0003}\\
            20 & \num{1,9824} & \num{1,9824} & \num{1,9824} & \num{-0,0001} & \num{0,0003} & \num{0,0001}\\
            21 & \num{1,9824} & \num{1,9824} & \num{1,9824} & \num{-0,0001} & \num{0,0001} & \num{-0,0000}\\
            22 & \num{1,9824} & \num{1,9824} & \num{1,9824} & \num{-0,0000} & \num{0,0001} & \num{0,0000}\\
            23 & \num{1,9824} & \num{1,9824} & \num{1,9824} & \num{-0,0000} & \num{0,0000} & \num{-0,0000}\\
            24 & \num{1,9824} & \num{1,9824} & \num{1,9824} & \num{-0,0000} & \num{0,0000} & \num{-0,0000}\\
             \hline
            \end{tabular} \\           
        Dengan demikian, akar persamaan dari $10^x = 100 - 2x$ hingga 4 digit di belakang koma adalah $x = \num{1,9824}$.

        \item $\ln{x} = 1 + 1/x^2$ ($x_0 = \num{3}$; s.d. 4 digit) \\
        \penyelesaian \\
        \begin{tabular}{|c|c|c|c|c|c|c|}
            \hline
            iterasi & $x_1$ & $x_2$ & $x_3$ & $f(x_1)$ & $f(x_2)$ & $f(x_3)$ \\
            \hline
            1 & \num{2,5000} & \num{3,5000} & \num{3,0000} & \num{-0,2437} & \num{0,1711} & \num{-0,0125}\\
            2 & \num{3,0000} & \num{3,5000} & \num{3,2500} & \num{-0,0125} & \num{0,1711} & \num{0,0840}\\
            3 & \num{3,0000} & \num{3,2500} & \num{3,1250} & \num{-0,0125} & \num{0,0840} & \num{0,0370}\\
            4 & \num{3,0000} & \num{3,1250} & \num{3,0625} & \num{-0,0125} & \num{0,0370} & \num{0,0126}\\
            5 & \num{3,0000} & \num{3,0625} & \num{3,0312} & \num{-0,0125} & \num{0,0126} & \num{0,0001}\\
            6 & \num{3,0000} & \num{3,0312} & \num{3,0156} & \num{-0,0125} & \num{0,0001} & \num{-0,0062}\\
            7 & \num{3,0156} & \num{3,0312} & \num{3,0234} & \num{-0,0062} & \num{0,0001} & \num{-0,0030}\\
            8 & \num{3,0234} & \num{3,0312} & \num{3,0273} & \num{-0,0030} & \num{0,0001} & \num{-0,0014}\\
            9 & \num{3,0273} & \num{3,0312} & \num{3,0293} & \num{-0,0014} & \num{0,0001} & \num{-0,0006}\\
            10 & \num{3,0293} & \num{3,0312} & \num{3,0303} & \num{-0,0006} & \num{0,0001} & \num{-0,0002}\\
            11 & \num{3,0303} & \num{3,0312} & \num{3,0308} & \num{-0,0002} & \num{0,0001} & \num{-0,0001}\\
            12 & \num{3,0308} & \num{3,0312} & \num{3,0310} & \num{-0,0001} & \num{0,0001} & \num{0,0000}\\
            13 & \num{3,0308} & \num{3,0310} & \num{3,0309} & \num{-0,0001} & \num{0,0000} & \num{-0,0000}\\
             \hline
            \end{tabular} \\      
        Dengan demikian, akar persamaan dari $\ln{x} = 1 + 1/x^2$ hingga 4 digit di belakang koma adalah $x = \num{3,0309}$.     

        \item $e^x - \ln{x} = 20$ ($x_0 = \num{3}$; s.d. 5 digit) \\ 
        \penyelesaian \\
        \begin{tabular}{|c|c|c|c|c|c|c|}
            \hline
            iterasi & $x_1$ & $x_2$ & $x_3$ & $f(x_1)$ & $f(x_2)$ & $f(x_3)$ \\
            \hline
            1 & \num{2,50000} & \num{3,50000} & \num{3,00000} & \num{-8,73380} & \num{11,86269} & \num{-1,01308}\\
            2 & \num{3,00000} & \num{3,50000} & \num{3,25000} & \num{-1,01308} & \num{11,86269} & \num{4,61168}\\
            3 & \num{3,00000} & \num{3,25000} & \num{3,12500} & \num{-1,01308} & \num{4,61168} & \num{1,62046}\\
            4 & \num{3,00000} & \num{3,12500} & \num{3,06250} & \num{-1,01308} & \num{1,62046} & \num{0,26171}\\
            5 & \num{3,00000} & \num{3,06250} & \num{3,03125} & \num{-1,01308} & \num{0,26171} & \num{-0,38585}\\
            6 & \num{3,03125} & \num{3,06250} & \num{3,04688} & \num{-0,38585} & \num{0,26171} & \num{-0,06465}\\
            7 & \num{3,04688} & \num{3,06250} & \num{3,05469} & \num{-0,06465} & \num{0,26171} & \num{0,09788}\\
            8 & \num{3,04688} & \num{3,05469} & \num{3,05078} & \num{-0,06465} & \num{0,09788} & \num{0,01645}\\
            9 & \num{3,04688} & \num{3,05078} & \num{3,04883} & \num{-0,06465} & \num{0,01645} & \num{-0,02414}\\
            10 & \num{3,04883} & \num{3,05078} & \num{3,04980} & \num{-0,02414} & \num{0,01645} & \num{-0,00386}\\
            11 & \num{3,04980} & \num{3,05078} & \num{3,05029} & \num{-0,00386} & \num{0,01645} & \num{0,00629}\\
            12 & \num{3,04980} & \num{3,05029} & \num{3,05005} & \num{-0,00386} & \num{0,00629} & \num{0,00122}\\
            13 & \num{3,04980} & \num{3,05005} & \num{3,04993} & \num{-0,00386} & \num{0,00122} & \num{-0,00132}\\
            14 & \num{3,04993} & \num{3,05005} & \num{3,04999} & \num{-0,00132} & \num{0,00122} & \num{-0,00005}\\
            15 & \num{3,04999} & \num{3,05005} & \num{3,05002} & \num{-0,00005} & \num{0,00122} & \num{0,00058}\\
            16 & \num{3,04999} & \num{3,05002} & \num{3,05000} & \num{-0,00005} & \num{0,00058} & \num{0,00027}\\
            17 & \num{3,04999} & \num{3,05000} & \num{3,05000} & \num{-0,00005} & \num{0,00027} & \num{0,00011}\\
            18 & \num{3,04999} & \num{3,05000} & \num{3,04999} & \num{-0,00005} & \num{0,00011} & \num{0,00003}\\
            19 & \num{3,04999} & \num{3,04999} & \num{3,04999} & \num{-0,00005} & \num{0,00003} & \num{-0,00001}\\
            20 & \num{3,04999} & \num{3,04999} & \num{3,04999} & \num{-0,00001} & \num{0,00003} & \num{0,00001}\\
            21 & \num{3,04999} & \num{3,04999} & \num{3,04999} & \num{-0,00001} & \num{0,00001} & \num{-0,00000}\\
            22 & \num{3,04999} & \num{3,04999} & \num{3,04999} & \num{-0,00000} & \num{0,00001} & \num{0,00000}\\
            23 & \num{3,04999} & \num{3,04999} & \num{3,04999} & \num{-0,00000} & \num{0,00000} & \num{0,00000}\\
            24 & \num{3,04999} & \num{3,04999} & \num{3,04999} & \num{-0,00000} & \num{0,00000} & \num{-0,00000}\\
             \hline
            \end{tabular} \\           
        Dengan demikian, akar persamaan dari $e^x - \ln{x} = 20$ hingga 5 digit di belakang koma adalah $x = \num{3,04999}$.  

        \item $10^x - 1$ ($x_0 = \num{0}$; s.d. 4 digit) \\ 
        \penyelesaian \\
        \begin{tabular}{|c|c|c|c|c|c|c|}
            \hline
            iterasi & $x_1$ & $x_2$ & $x_3$ & $f(x_1)$ & $f(x_2)$ & $f(x_3)$ \\
            \hline
            1 & \num{-0,5000} & \num{1,5000} & \num{0,5000} & \num{-0,6838} & \num{30,6228} & \num{2,1623}\\
            2 & \num{-0,5000} & \num{0,5000} & \num{0,0000} & \num{-0,6838} & \num{2,1623} & \num{0,0000}\\
             \hline
            \end{tabular} \\
        Dengan demikian, akar persamaan $10^x - 1$ hingga 4 digit di belakang koma adalah $x = \num{0,0000}$;
                        
    \end{enumerate}
\end{enumerate}

\end{document}