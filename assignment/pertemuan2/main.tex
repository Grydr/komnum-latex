\documentclass{article}
\usepackage[utf8]{inputenc}
\usepackage[a4paper, margin=1in]{geometry}
\usepackage{siunitx}
\usepackage{amsmath}
\usepackage{enumitem}
\usepackage{esdiff}

\tolerance=1
\emergencystretch=\maxdimen
\hyphenpenalty=10000
\hbadness=10000

\sisetup{
    input-ignore={.},
    output-decimal-marker={,},
    group-minimum-digits=4,
    group-separator={.},
    group-digits=integer
}

\newcommand{\penyelesaian}{\textbf{Penyelesaian: }}

\title{\textbf{Komputasi Numerik: Tugas 2}}
\author{Kelompok 15}
\date{}

\begin{document}

\maketitle

\begin{enumerate}
    \item Buatlah suatu analisa mengenai metode yang memiliki tingkat akurasi \& presisi yang paling tinggi dalam menyelesaikan persamaan berikut : \\
    \ $f(x) = \frac{(1 - 0,6x)}{x}$ \\
    \ perhitungan dibuat sampai 3 iterasi dengan $x_0$ = 2.
    \begin{enumerate}
        \item Metode Tabulasi: \\
        \penyelesaian   \[
                        \begin{array}{|c|c|}
                        \hline
                        x & f(x) = \frac{1 - 0.6x}{x} \\
                        \hline
                        1.0 & \frac{1 - 0.6(1.0)}{1.0} = 0.4 \\
                        1.5 & \frac{1 - 0.6(1.5)}{1.5} = 0.2 \\
                        2.0 & \frac{1 - 0.6(2.0)}{2.0} = -0.1 \\
                        2.5 & \frac{1 - 0.6(2.5)}{2.5} = -0.24 \\
                        \hline
                        \end{array}
                        \]

        \item Metode Bolzano: \\
        \penyelesaian   \[
                        \begin{array}{|c|c|c|}
                        \hline
                        x & f(x) = \frac{1 - 0.6x}{x} & \text{Tanda} \\
                        \hline
                        1.5 & \frac{1 - 0.6(1.5)}{1.5} = 0.2 & + \\
                        2.0 & \frac{1 - 0.6(2.0)}{2.0} = -0.1 & - \\
                        \hline
                        \end{array}
                        \]

        \item Metode Regula Falsi: \\
        \penyelesaian   \[
                        \begin{array}{|c|c|c|}
                        \hline
                        \text{Iterasi} & x & f(x) = \frac{1 - 0.6x}{x} \\
                        \hline
                        1 & 0 & 2.0 \\
                        2 & 1 & 0.4 \\
                        3 & 1.25 & 0.2 \\
                        4 & 1.5 & 0.066667 \\
                        \hline
                        \end{array}
                        \]

    \end{enumerate}

\end{enumerate}

\end{document}