\documentclass{article}
\usepackage[utf8]{inputenc}
\usepackage[a4paper, margin=1in]{geometry}
\usepackage{siunitx}
\usepackage{amsmath}
\usepackage{enumitem}
\usepackage{esdiff}

\tolerance=1
\emergencystretch=\maxdimen
\hyphenpenalty=10000
\hbadness=10000

\sisetup{
    input-ignore={.},
    output-decimal-marker={,},
    group-minimum-digits=4,
    group-separator={.},
    group-digits=integer
}

\newcommand{\penyelesaian}{\textbf{Penyelesaian: }}

\title{\textbf{Komputasi Numerik: Tugas 2}}
\author{Kelompok 15}
\date{}

\begin{document}

\maketitle

\begin{enumerate}
    \item Soal 5
    \begin{enumerate}
        \item Subsoal (a) \\
        \penyelesaian Penyelesaian soal 1 subsoal (a)

        \item Subsoal (b) \\
        \penyelesaian Penyelesaian soal 1 subsoal (b)

        \item Subsoal (c) \\
        \penyelesaian  Penyelesaian soal 1 subsoal (c)
    \end{enumerate}

    \item Soal 2
    \penyelesaian Penyelesaian soal 2
\end{enumerate}

\end{document}