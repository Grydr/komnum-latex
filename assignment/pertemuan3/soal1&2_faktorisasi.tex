\documentclass{article}
\usepackage[bahasa]{babel}
\usepackage{amsmath}
\usepackage{booktabs}
\usepackage{array}
\usepackage{caption}

\begin{document}

\section*{Metode Faktorisasi}
\subsection*{A. Persamaan :}

\begin{equation}
    P_3(x) = x^3 + 6.6x^2 - 29.05x + 22.64 = 0
\end{equation}

\subsubsection*{Bentuk Faktorisasi:}
\begin{equation}
    P_3(x) = (x + b_0)(x^2 + a_1x + a_0)
\end{equation}

\subsubsection*{Tabel Iterasi:}
\begin{table}[h]
\centering
\caption{Proses Iterasi Metode Faktorisasi (Kubik)}
\begin{tabular}{|c|r|r|r|l|}
\hline
\textbf{Iterasi} & \multicolumn{1}{c|}{\textbf{b\textsubscript{0}}} & \multicolumn{1}{c|}{\textbf{a\textsubscript{1}}} & \multicolumn{1}{c|}{\textbf{a\textsubscript{0}}} & \textbf{Keterangan} \\ \hline
1 & 0.000 & 6.600 & -29.050 & Inisialisasi ($b_0 = 0$) \\ \hline
2 & -0.779 & 7.379 & -23.300 & $b_0 = 22.64/(-29.05)$ \\ \hline
3 & -0.972 & 7.572 & -21.690 & $b_0 = 22.64/(-23.30)$ \\ \hline
4 & -1.044 & 7.644 & -21.070 & $b_0 = 22.64/(-21.69)$ \\ \hline
5 & -1.075 & 7.675 & -20.800 & $b_0 = 22.64/(-21.07)$ \\ \hline
6 & -1.088 & 7.688 & -20.800 &  \\ \hline
\end{tabular}
\end{table}

\subsubsection*{Hasil Faktorisasi:}
\begin{equation}
    P_3(x) \approx (x - 1.088)(x^2 + 7.688x - 20.8)
\end{equation}

\subsubsection*{Akar-Akar Persamaan:}
\begin{enumerate}
    \item \textbf{Akar Faktor Linear}:
    \begin{equation}
        x - 1.088 = 0 \implies x \approx \boxed{1.088}
    \end{equation}
    
    \item \textbf{Akar Faktor Kuadrat}:
    \begin{equation}
        x = \frac{-7.688 \pm \sqrt{7.688^2 - 4 \times (-20.8)}}{2}
    \end{equation}
    \begin{equation}
        x \approx \boxed{2.121} \quad \text{dan} \quad \boxed{-9.809}
    \end{equation}
\end{enumerate}

\subsection*{B. Persamaan :}
\begin{equation}
    P_4(x) = x^4 - 0.41x^3 + 1.632x^2 - 9.146x + 7.260 = 0
\end{equation}

\subsubsection*{Bentuk Faktorisasi:}
\begin{equation}
    P_4(x) = (x^2 + b_1x + b_0)(x^2 + a_1x + a_0)
\end{equation}

\subsubsection*{Tabel Iterasi:}
\begin{table}[h]
\centering
\caption{Proses Iterasi Metode Faktorisasi (Kuartik)}
\begin{tabular}{|c|r|r|r|r|l|}
\hline
\textbf{Iterasi} & \multicolumn{1}{c|}{\textbf{b\textsubscript{0}}} & \multicolumn{1}{c|}{\textbf{b\textsubscript{1}}} & \multicolumn{1}{c|}{\textbf{a\textsubscript{1}}} & \multicolumn{1}{c|}{\textbf{a\textsubscript{0}}} & \textbf{Keterangan} \\ \hline
1 & 0.000 & 0.000 & 1.632 & -9.146 & Inisialisasi ($b_0 = 0$) \\ \hline
2 & -0.794 & 1.022 & 2.426 & -7.260 & $b_0 = 7.260/(-9.146)$ \\ \hline
3 & -1.012 & 1.305 & 2.644 & -6.678 & $b_0 = 7.260/(-7.260)$ \\ \hline
4 & -1.101 & 1.447 & 2.733 & -6.400 & $b_0 = 7.260/(-6.678)$ \\ \hline
5 & -1.145 & 1.518 & 2.777 & -6.260 & $b_0 = 7.260/(-6.400)$ \\ \hline
6 & -1.167 & 1.553 & 2.799 & -6.200 & Konvergensi tercapai \\ \hline
\end{tabular}
\end{table}

\subsubsection*{Hasil Faktorisasi:}
\begin{equation}
    P_4(x) \approx (x^2 + 1.553x - 1.167)(x^2 + 2.799x + 6.200)
\end{equation}

\subsubsection*{Akar-Akar Persamaan:}
\begin{enumerate}
    \item \textbf{Faktor Pertama}:
    \begin{equation}
        x = \frac{-1.553 \pm \sqrt{(1.553)^2 - 4 \times (-1.167)}}{2}
    \end{equation}
    \begin{equation}
        x \approx \boxed{0.577} \quad \text{dan} \quad \boxed{-2.130}
    \end{equation}
    
    \item \textbf{Faktor Kedua}:
    \begin{equation}
        x = \frac{-2.799 \pm \sqrt{(2.799)^2 - 4 \times 6.200}}{2}
    \end{equation}
    \begin{equation}
        x \approx \boxed{-1.400 + 1.825i} \quad \text{dan} \quad \boxed{-1.400 - 1.825i}
    \end{equation}
\end{enumerate}

\end{document}