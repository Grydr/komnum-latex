\documentclass{article}
\usepackage[utf8]{inputenc}
\usepackage[a4paper, margin=1in]{geometry}
\usepackage{siunitx}
\usepackage{amsmath}
\usepackage{enumitem}
\usepackage{esdiff}
\usepackage{pgfplots}
\usepackage{listings}
\usepackage{xcolor}

\pgfplotsset{compat=1.18, width=10cm}

\tolerance=1
\emergencystretch=\maxdimen
\hyphenpenalty=10000
\hbadness=10000

\sisetup{
    input-ignore={.},
    output-decimal-marker={,},
    group-minimum-digits=4,
    group-separator={.},
    group-digits=integer
}

\definecolor{darkgray}{rgb}{.4,.4,.4}

\lstdefinestyle{code}{
    aboveskip={1.3\baselineskip},
    basicstyle=\normalsize\ttfamily\linespread{4},
    breaklines=false,
    columns=fullflexible,
    commentstyle=\color[rgb]{0.127,0.427,0.514}\ttfamily\itshape,
    escapechar=@,
    extendedchars=true,
    frame=single,
    identifierstyle=\color{black},
    inputencoding=latin1,
    keywordstyle=\color[HTML]{228B22}\bfseries,
    ndkeywordstyle=\color[HTML]{228B22}\bfseries,
    numbers=left,
    numberstyle=\normalsize,
    prebreak = \raisebox{0ex}[0ex][0ex]{\ensuremath{\hookleftarrow}},
    stringstyle=\color[rgb]{0.639,0.082,0.082}\ttfamily,
    upquote=true,
    showstringspaces=false,
    xleftmargin=5ex,
    aboveskip=5pt
}

\newcommand{\penyelesaian}{\textbf{Penyelesaian: }}

\title{\textbf{Komputasi Numerik: Tugas 3}}
\author{Kelompok 15}
\date{}

\begin{document}

\maketitle

\begin{enumerate}
    \item Dapatkan akar-akar persamaan berikut: 
    \begin{enumerate}
        \item $x^3 + \num{6,6}x^2 - \num{29,05}x + \num{22,64} = 0$
        \item $x^4 - \num{0,41}x^3 + \num{1,632}x^2 - \num{9,146}x + \num{7,260} = 0$
    \end{enumerate}
    Dengan metode Iterasi. \\
    \penyelesaian Penyelesaian soal 1

    \item Dapatkan akar-akar persamaan berikut: 
    \begin{enumerate}
        \item $x^3 + \num{6,6}x^2 - \num{29,05}x + \num{22,64} = 0$
        \item $x^4 - \num{0,41}x^3 + \num{1,632}x^2 - \num{9,146}x + \num{7,260} = 0$
    \end{enumerate}
    Dengan metode Faktorisasi. \\
    \penyelesaian Penyelesaian soal 2

    \item Gunakan metode Newton-Raphson untuk mendapatkan akar persamaan: \\
    \begin{equation*}
        f(x) = -\num{0,875}x^2 + \num{1,75}x + \num{2,625}
    \end{equation*}
    dengan $x_i = \num{3,1}$ \\
    \penyelesaian

    \item Gunakan metode Newton-Raphson untuk mendapatkan akar persamaan: \\
    \begin{equation*}
        f(x) = -\num{2,1} + \num{6,21}x - \num{3,9}x^2 + \num{0,667}x^3
    \end{equation*}
    \penyelesaian

    \item Gunakan metode Newton-Raphson untuk mendapatkan akar persamaan: \\
    \begin{equation*}
        f(x) = -\num{23,33} + \num{79,35}x - \num{88,09}x^2 + \num{41,6}x^3 - \num{8,68}x^4 + \num{0,658}x^5
    \end{equation*}
    dengan $x_i = \num{3,5}$ \\
    \penyelesaian

    \item Gunakan metode Secant untuk mendapatkan akar dari persamaan:
    \begin{equation*}
        f(x) = \num{9,36} - \num{21,963}x + \num{16,2965}x^2 - \num{3,70377}x^3
    \end{equation*}
    \penyelesaian Dengan $x_{i-1} = 0$ dan $x_i = 1$, maka: \\
    $f(x_{i-1}) = f(0) = \num{9,36} - \num{21,963}(0) + \num{16,2965}(0)^2 - \num{3,70377}(0)^3 = \num{9,36}$ dan \\ 
    $f(x_{i}) = f(1) = \num{9,36} - \num{21,963}(1) + \num{16,2965}(1)^2 - \num{3,70377}(1)^3 = \num{-0,01027}$. \\

    Menggunakan metode Secant, dapat diperoleh hasil iterasi pertama:
    \begin{align*}
        x_{i+1} &= x_i - \frac{f(x_i)(x_{i-1} - x_i)}{f(x_{i-1}) - f(x_i)} \\
        &= 1 - \frac{(-\num{0,01027})(-1)}{\num{9,36} - \num{0,01027}} \\
        &= \num{0,99890}
    \end{align*}

    Dengan iterasi berikutnya hingga nilai $|f(x_i)|$ mendekati nol, diperoleh hasil sebagai berikut.\\
    \begin{tabular}{|c|c|c|c|c|c|}
        \hline
        iterasi & $x_{i-1}$ & $x_i$ & $f(x_{i-1})$ & $f(x_i)$ & $x_{i+1}$ \\
        \hline
        1 & \num{0,00000} & \num{1,00000} & \num{9,36000} & \num{-0,01027} & \num{0,99890}\\
        2 & \num{1,00000} & \num{0,99890} & \num{-0,01027} & \num{-0,00974} & \num{0,97891}\\
        3 & \num{0,99890} & \num{0,97891} & \num{-0,00974} & \num{0,00222} & \num{0,98262}\\
        4 & \num{0,97891} & \num{0,98262} & \num{0,00222} & \num{-0,00032} & \num{0,98215}\\
        5 & \num{0,98262} & \num{0,98215} & \num{-0,00032} & \num{-0,00001} & \num{0,98214}\\
        6 & \num{0,98215} & \num{0,98214} & \num{-0,00001} & \num{0,00000} & \num{0,98214}\\
         \hline
        \end{tabular} \\
    Dengan demikian, akar dari $f(x)$ adalah $x \approx \num{0,98214}$.

    \item Gunakan metode Secant untuk mendapatkan akar dari persamaan:
    \begin{equation*}
        f(x) = x^4 - \num{8,6}x^3 - \num{35,51}x^2 + \num{464}x - \num{998,46}
    \end{equation*}
    dengan $x_{i-1} = \num{7}$ dan $x_i = \num{9}$ \\
    \penyelesaian Dengan \\
    $f(x_{i-1}) = f(7) = 7^4 - \num{8,6}(7)^3 - \num{35,51}(7)^2 + \num{464}(7) - \num{998,46} = \num{-39,25}$ dan \\
    $f(x_i) = f(9) = 9^4 - \num{8,6}(9)^3 - \num{35,51}(9)^2 + \num{464}(9) - \num{998,46} = \num{592,83}$, \\
    Menggunakan metode Secant, dapat diperoleh hasil iterasi pertama:
    \begin{align*}
        x_{i+1} &= x_i - \frac{f(x_i)(x_{i-1} - x_i)}{f(x_{i-1}) - f(x_i)} \\
        &= 9 - \frac{(\num{529,83})(7 - 9)}{\num{-39,25} - \num{592,83}} \\
        &= \num{7,12419}
    \end{align*}

    Dengan iterasi berikutnya hingga nilai $|f(x_i)|$ mendekati nol, diperoleh hasil sebagai berikut.\\
    \begin{tabular}{|c|c|c|c|c|c|}
        \hline
        iterasi & $x_{i-1}$ & $x_i$ & $f(x_{i-1})$ & $f(x_i)$ & $x_{i+1}$ \\
        \hline
        1 & \num{7,00000} & \num{9,00000} & \num{-39,25000} & \num{592,83000} & \num{7,12419}\\
        2 & \num{9,00000} & \num{7,12419} & \num{592,83000} & \num{-28,73897} & \num{7,21092}\\
        3 & \num{7,12419} & \num{7,21092} & \num{-28,73897} & \num{-19,85323} & \num{7,40470}\\
        4 & \num{7,21092} & \num{7,40470} & \num{-19,85323} & \num{5,03497} & \num{7,36550}\\
        5 & \num{7,40470} & \num{7,36550} & \num{5,03497} & \num{-0,59129} & \num{7,36962}\\
        6 & \num{7,36550} & \num{7,36962} & \num{-0,59129} & \num{-0,01458} & \num{7,36972}\\
        7 & \num{7,36962} & \num{7,36972} & \num{-0,01458} & \num{0,00004} & \num{7,36972}\\
        8 & \num{7,36972} & \num{7,36972} & \num{0,00004} & \num{-0,00000} & \num{7,36972}\\
         \hline
        \end{tabular} \\
    Dengan demikian, akar dari $f(x)$ adalah $x \approx \num{7,36972}$.

    \item Gunakan metode Secant untuk mendapatkan akar dari persamaan:
    \begin{equation*}
        f(x) = x^3 - \num{6}x^2 + \num{11}x - \num{6}
    \end{equation*}
    dengan $x_{i-1} = \num{2,5}$ dan $x_i = \num{3,6}$ \\
    \penyelesaian Dengan \\
    $f(x_{i-1}) = f(2,5) = (2,5)^3 - \num{6}(2,5)^2 + \num{11}(2,5) - \num{6} = \num{-0,375}$ dan \\
    $f(x_i) = f(3,6) = (3,6)^3 - \num{6}(3,6)^2 + \num{11}(3,6) - \num{6} = \num{2,496}$. \\
    Menggunakan metode Secant, dapat diperoleh hasil iterasi pertama:
    \begin{align*}
        x_{i+1} &= x_i - \frac{f(x_i)(x_{i-1} - x_i)}{f(x_{i-1}) - f(x_i)} \\
        &= \num{3,6} - \frac{(\num{2,496})(\num{2,5} - \num{3,6})}{\num{-0,375} - \num{2,496}} \\
        &= \num{2,64368}.
    \end{align*}

    Dengan iterasi berikutnya hingga nilai $|f(x_i)|$ mendekati nol, diperoleh hasil sebagai berikut.\\
    \begin{tabular}{|c|c|c|c|c|c|}
        \hline
        iterasi & $x_{i-1}$ & $x_i$ & $f(x_{i-1})$ & $f(x_i)$ & $x_{i+1}$ \\
        \hline
        1 & \num{2,50000} & \num{3,60000} & \num{-0,37500} & \num{2,49600} & \num{2,64368}\\
        2 & \num{3,60000} & \num{2,64368} & \num{2,49600} & \num{-0,37699} & \num{2,76917}\\
        3 & \num{2,64368} & \num{2,76917} & \num{-0,37699} & \num{-0,31412} & \num{3,39610}\\
        4 & \num{2,76917} & \num{3,39610} & \num{-0,31412} & \num{1,32505} & \num{2,88931}\\
        5 & \num{3,39610} & \num{2,88931} & \num{1,32505} & \num{-0,18598} & \num{2,95169}\\
        6 & \num{2,88931} & \num{2,95169} & \num{-0,18598} & \num{-0,08974} & \num{3,00985}\\
        7 & \num{2,95169} & \num{3,00985} & \num{-0,08974} & \num{0,01999} & \num{2,99925}\\
        8 & \num{3,00985} & \num{2,99925} & \num{0,01999} & \num{-0,00149} & \num{2,99999}\\
        9 & \num{2,99925} & \num{2,99999} & \num{-0,00149} & \num{-0,00002} & \num{3,00000}\\
        10 & \num{2,99999} & \num{3,00000} & \num{-0,00002} & \num{0,00000} & \num{3,00000}\\
         \hline
        \end{tabular} \\
    Dengan demikian, akar dari $f(x)$ adalah $x \approx 3$.

    \item Buatlah sebuah paparan untuk menjelaskan tentang metode Bairstow dan metode Quotient-Difference (Q-D). 
    Dan buatlah sebuah kesimpulan mengenai kemudahan/kesulitan kedua metode tersebut didalam menyelesaikan masalah dibanding dengan metode2 yang telah anda pelajari dalam materi ini. \\
    \penyelesaian
\end{enumerate}

\end{document}