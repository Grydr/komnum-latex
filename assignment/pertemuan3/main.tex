\documentclass{article}
\usepackage[utf8]{inputenc}
\usepackage[a4paper, margin=1in]{geometry}
\usepackage{siunitx}
\usepackage{amsmath}
\usepackage{amssymb}
\usepackage{enumitem}
\usepackage{esdiff}
\usepackage{pgfplots}
\usepackage{listings}
\usepackage{xcolor}
\usepackage{booktabs}

\pgfplotsset{compat=1.18, width=10cm}

\tolerance=1
\emergencystretch=\maxdimen
\hyphenpenalty=10000
\hbadness=10000

\sisetup{
    input-ignore={.},
    output-decimal-marker={,},
    group-minimum-digits=4,
    group-separator={.},
    group-digits=integer
}

\definecolor{darkgray}{rgb}{.4,.4,.4}

\lstdefinestyle{code}{
    aboveskip={1.3\baselineskip},
    basicstyle=\normalsize\ttfamily\linespread{4},
    breaklines=false,
    columns=fullflexible,
    commentstyle=\color[rgb]{0.127,0.427,0.514}\ttfamily\itshape,
    escapechar=@,
    extendedchars=true,
    frame=single,
    identifierstyle=\color{black},
    inputencoding=latin1,
    keywordstyle=\color[HTML]{228B22}\bfseries,
    ndkeywordstyle=\color[HTML]{228B22}\bfseries,
    numbers=left,
    numberstyle=\normalsize,
    prebreak = \raisebox{0ex}[0ex][0ex]{\ensuremath{\hookleftarrow}},
    stringstyle=\color[rgb]{0.639,0.082,0.082}\ttfamily,
    upquote=true,
    showstringspaces=false,
    xleftmargin=5ex,
    aboveskip=5pt
}

\newcommand{\penyelesaian}{\textbf{Penyelesaian: }}

\title{\textbf{Komputasi Numerik: Tugas 3}}
\author{Kelompok 15}
\date{}

\begin{document}

\maketitle

\begin{enumerate}
    \item Dapatkan akar-akar persamaan berikut dengan metode Iterasi: 
    \begin{enumerate}
        \item $x^3 + \num{6,6}x^2 - \num{29,05}x + \num{22,64} = 0$ \\
        \penyelesaian Untuk melakukan metode Iterasi, $f(x) = 0$ harus diubah menjadi $x = g(x)$:
        \begin{align*}
            x^3 + \num{6,6}x^2 - \num{29,05}x + \num{22,64} = 0 \\
            \implies x = \frac{x^3 + \num{6,6}x^2 + \num{22,64}}{\num{29,05}}
        \end{align*}
        Dengan demikian, 
        \begin{align*}
            g(x) = \frac{x^3 + \num{6,6}x^2 + \num{22,64}}{\num{29,05}}    
        \end{align*}

        Dengan asumsi $x_0 = 1$, selanjutnya akan dicari tahu apakah nilai $x_0$ tersebut memenuhi syarat $|g'(x_0)| < 1$:
        \begin{align*}
            \left|g'(x_0)\right| &< 1 \\
            \left|\frac{3(1)^2 + \num{13,2}(1)}{\num{29,05}}\right| &< 1 \\
            \left|\frac{\num{16,2}}{\num{29,05}}\right| &< 1 \\
            \num{0,55} &< 1
        \end{align*}
        Dengan demikian, dengan $x_0 = 1$, proses iterasi akan konvergen. \\

        Dengan relasi berulang:
        \begin{align*}
            x_{n+1} = \frac{{x_n}^3 + \num{6,6}{x_n}^2 + \num{22,64}}{\num{29,05}}
        \end{align*}
        diperoleh hasil proses iterasi sebagai berikut: \\
        \begin{tabular}{|c|c|c|}
            \hline
            n & $x_n$ & $x_{n+1}$ \\
            \hline
            0 & \num{1,000} & $\dfrac{\num{1,000}^3 + \num{6,6} \times \num{1,000}^2 + \num{22,64}}{\num{29,05}} = \num{1,040}$ \\
            1 & \num{1,040} & $\dfrac{\num{1,040}^3 + \num{6,6} \times \num{1,040}^2 + \num{22,64}}{\num{29,05}} = \num{1,065}$ \\
            2 & \num{1,065} & $\dfrac{\num{1,065}^3 + \num{6,6} \times \num{1,065}^2 + \num{22,64}}{\num{29,05}} = \num{1,080}$ \\
            3 & \num{1,080} & $\dfrac{\num{1,080}^3 + \num{6,6} \times \num{1,080}^2 + \num{22,64}}{\num{29,05}} = \num{1,090}$ \\
            4 & \num{1,090} & $\dfrac{\num{1,090}^3 + \num{6,6} \times \num{1,090}^2 + \num{22,64}}{\num{29,05}} = \num{1,095}$ \\
            5 & \num{1,095} & $\dfrac{\num{1,095}^3 + \num{6,6} \times \num{1,095}^2 + \num{22,64}}{\num{29,05}} = \num{1,097}$ \\
            \hline
        \end{tabular} \\
        Dengan demikian, akar dari persamaan di atas adalah $x \approx \num{1,097}$.
        
        \item $x^4 - \num{0,41}x^3 + \num{1,632}x^2 - \num{9,146}x + \num{7,260} = 0$ \\
        \penyelesaian $f(x) = 0 \iff x = g(x)$:
        \begin{align*}
            x^4 - \num{0,41}x^3 + \num{1,632}x^2 - \num{9,146}x + \num{7,260} = 0 \\
            \implies x = \frac{x^4 - \num{0,41}x^3 + \num{1,632}x^2 + \num{7,260}}{\num{9,146}} \\
            \therefore g(x) = \frac{x^4 - \num{0,41}x^3 + \num{1,632}x^2 + \num{7,260}}{\num{9,146}}
        \end{align*}
    
        Dengan asumsi $x_0 = 1$, selanjutnya akan dicari tahu apakah nilai $x_0$ tersebut memenuhi syarat $|g'(x_0)| < 1$:
        \begin{align*}
            \left|g'(x_0)\right| &< 1 \\
            \left|\frac{4(1)^3 - \num{1,23}(1)^2 + \num{3,264}(1) - \num{9,146}}{\num{9,146}}\right| &< 1 \\
            \left|\frac{\num{3,112}}{\num{9,146}}\right| &< 1 \\
            \num{0,34} &< 1
        \end{align*}
        Dengan demikian, dengan $x_0 = 1$, proses iterasi akan konvergen. \\
    
        Dengan relasi berulang:
        \begin{align*}
            x_{n+1} = \frac{{x_n}^4 - \num{0,41}{x_n}^3 + \num{1,632}{x_n}^2 + \num{7,260}}{\num{9,146}}
        \end{align*}
        diperoleh hasil proses iterasi sebagai berikut: \\
        \begin{tabular}{|c|c|c|}
            \hline
            n & $x_n$ & $x_{n+1}$ \\
            \hline
            0 & \num{1,000} & $\dfrac{\num{1,000}^4 - \num{0,41} \times \num{1,000}^3 + \num{1,632} \times \num{1,000}^2 + \num{7,260}}{\num{9,146}} = \num{1,037}$ \\
            1 & \num{1,037} & $\dfrac{\num{1,037}^4 - \num{0,41} \times \num{1,037}^3 + \num{1,632} \times \num{1,037}^2 + \num{7,260}}{\num{9,146}} = \num{1,062}$ \\
            2 & \num{1,062} & $\dfrac{\num{1,062}^4 - \num{0,41} \times \num{1,062}^3 + \num{1,632} \times \num{1,062}^2 + \num{7,260}}{\num{9,146}} = \num{1,078}$ \\
            3 & \num{1,078} & $\dfrac{\num{1,078}^4 - \num{0,41} \times \num{1,078}^3 + \num{1,632} \times \num{1,078}^2 + \num{7,260}}{\num{9,146}} = \num{1,088}$ \\
            4 & \num{1,088} & $\dfrac{\num{1,088}^4 - \num{0,41} \times \num{1,088}^3 + \num{1,632} \times \num{1,088}^2 + \num{7,260}}{\num{9,146}} = \num{1,094}$ \\
            5 & \num{1,094} & $\dfrac{\num{1,094}^4 - \num{0,41} \times \num{1,094}^3 + \num{1,632} \times \num{1,094}^2 + \num{7,260}}{\num{9,146}} = \num{1,098}$ \\
            \hline
        \end{tabular} \\
        Dengan demikian, akar dari persamaan di atas adalah $x \approx \num{1,098}$.
    \end{enumerate}

    \item Dapatkan akar-akar persamaan berikut dengan metode Faktorisasi: 
    \begin{enumerate}
        \item $x^3 + \num{6,6}x^2 - \num{29,05}x + \num{22,64} = 0$ \\
        \penyelesaian Persamaan tersebut memenuhi rumusan metode Faktorisasi derajat tiga.
        Dengan $P_3(x) = x^3 + A_2x^2 + A_1x + A_0$, bentuk faktornya adalah $P_3(x) = (x + b_0)(x^2 + a_1x + a_0)$. \\

        Dengan $P_3(x) = x^3 + \num{6,6}x^2 - \num{29,05}x + \num{22,64} = 0$ dan fakta bahwa $b_0 = A_0/a_0$, hasil proses iterasi diperoleh sebagai berikut: \\
        \begin{tabular}{|c|r|r|r|l|}
            \hline
            \textbf{iterasi} & \multicolumn{1}{c|}{\textbf{b\textsubscript{0}}} & \multicolumn{1}{c|}{\textbf{a\textsubscript{1}}} & \multicolumn{1}{c|}{\textbf{a\textsubscript{0}}} & \textbf{keterangan} \\ \hline
            1 & \num{0,000} & \num{6,600} & \num{-29,050} & Inisialisasi ($b_0 = 0$) \\ \hline
            2 & \num{-0,779} & \num{7,379} & \num{-23,300} & $b_0 = \num{22,64}/(\num{-29,05})$ \\ \hline
            3 & \num{-0,972} & \num{7,572} & \num{-21,690} & $b_0 = \num{22,64}/(\num{-23,30})$ \\ \hline
            4 & \num{-1,044} & \num{7,644} & \num{-21,070} & $b_0 = \num{22,64}/(\num{-21,69})$ \\ \hline
            5 & \num{-1,075} & \num{7,675} & \num{-20,800} & $b_0 = \num{22,64}/(\num{-21,07})$ \\ \hline
            6 & \num{-1,088} & \num{7,688} & \num{-20,800} &  \\ \hline
        \end{tabular} \\
        Dengan demikian, diperoleh hasil faktorisasi: $P_3(x) \approx (x - \num{1,088})(x^2 + \num{7,688}x - \num{20,800})$. \\
        
        Akar-akar dari persamaan tersebut antara lain:
        \begin{itemize}
            \item $(x - \num{1,088}) = 0 \implies x \approx \num{1,088}$.
            \item $(x^2 + \num{7,688}x - \num{20,800}) = 0$ \\
            $ \implies x = \frac{-\num{7,688} \pm \sqrt{\num{7,688}^2 - 4 \times \num{-20,800}}}{2}$ \\
            $ \therefore x \approx \num{2,121} \vee x \approx \num{-9,809}$.
        \end{itemize} 
        Dengan demikian, terdapat tiga akar berbeda, yakni $x_1 = \num{1,088}$, $x_2 = \num{2,121}$, dan $x_3 = \num{-9,809}$.


        \item $x^4 - \num{0,41}x^3 + \num{1,632}x^2 - \num{9,146}x + \num{7,260} = 0$ \\
        \penyelesaian Persamaan tersebut memenuhi rumusan metode faktorisasi derajat empat.
        Dengan $P_4(x) = x^4 + A_3x^3 + A_2x^2 + A_1x + A_0$, bentuk faktornya adalah $P_4(x) = (x^2 + b_1x + b_0)(x^2 + a_1x + a_0)$. \\
        
        Dengan $P_4(x) = x^4 - \num{0,41}x^3 + \num{1,632}x^2 - \num{9,146}x + \num{7,260} = 0$, hasil proses iterasi diperoleh sebagai berikut: \\
        \begin{tabular}{|c|r|r|r|r|l|}
            \hline
            \textbf{iterasi} & \multicolumn{1}{c|}{\textbf{b\textsubscript{0}}} & \multicolumn{1}{c|}{\textbf{b\textsubscript{1}}} & \multicolumn{1}{c|}{\textbf{a\textsubscript{1}}} & \multicolumn{1}{c|}{\textbf{a\textsubscript{0}}} & \textbf{keterangan} \\ \hline
            1 & \num{0,000} & \num{0,000} & \num{1,632} & \num{-9,146} & Inisialisasi ($b_0 = 0$) \\ \hline
            2 & \num{-0,794} & \num{1,022} & \num{2,426} & \num{-7,260} & $b_0 = \num{7,260}/(\num{-9,146})$ \\ \hline
            3 & \num{-1,012} & \num{1,305} & \num{2,644} & \num{-6,678} & $b_0 = \num{7,260}/(\num{-7,260})$ \\ \hline
            4 & \num{-1,101} & \num{1,447} & \num{2,733} & \num{-6,400} & $b_0 = \num{7,260}/(\num{-6,678})$ \\ \hline
            5 & \num{-1,145} & \num{1,518} & \num{2,777} & \num{-6,260} & $b_0 = \num{7,260}/(\num{-6,400})$ \\ \hline
            6 & \num{-1,167} & \num{1,553} & \num{2,799} & \num{-6,200} & Konvergensi tercapai \\ \hline
        \end{tabular} \\
        Dengan demikian, diperoleh hasil faktorisasi:\\
        $P_4(x) \approx (x^2 + \num{1,553}x - \num{1,167})(x^2 + \num{2,799}x + \num{6,200})$. \\
        
        Akar-akar dari persamaan tersebut antara lain:
        \begin{itemize}
            \item $(x^2 + \num{1,553}x - \num{1,167}) = 0$ \\
            $ \implies x = \frac{-\num{1,553} \pm \sqrt{(\num{1,553})^2 - 4 \times \num{-1,167}}}{2}$ \\
            $ \therefore x \approx \num{0,577} \vee x \approx \num{-2,130}$.
        
            \item $(x^2 + \num{2,799}x + \num{6,200}) = 0$ \\
            $ \implies x = \frac{-\num{2,799} \pm \sqrt{(\num{2,799})^2 - 4 \times \num{6,200}}}{2}$ \\
            $ \therefore x \approx \num{-1,400} + \num{1,825}i \vee x \approx \num{-1,400} - \num{1,825}i$.
        \end{itemize} 
        
        Dengan demikian, terdapat empat akar berbeda, yakni $x_1 = \num{0,577}$, $x_2 = \num{-2,130}$, $x_3 = \num{-1,400} + \num{1,825}i$, dan $x_4 = \num{-1,400} - \num{1,825}i$.
    \end{enumerate}

    \item Gunakan metode Newton-Raphson untuk mendapatkan akar persamaan: \\
    \begin{equation*}
        f(x) = -\num{0,875}x^2 + \num{1,75}x + \num{2,625}
    \end{equation*}
    dengan $x_i = \num{3,1}$ \\
    \penyelesaian

    Diketahui: \\
    \begin{equation*}
    \begin{split}
        x_i & = \num{3,1} \\
        f(x) & = -\num{0,875}x^2 + \num{1,75}x + \num{2,625} \\
        f'(x) & = -\num{1,75}x + \num{1,75} \\ 
    \end{split}
    \end{equation*}

    Nilai awal: \\
    \begin{equation*}
    \begin{split}
        f(\num{3,1}) & = -\num{0,875}(\num{3,1})^2 + \num{1,75}(\num{3,1}) + \num{2,625} = -\num{0,35875} \\
        f'(\num{3,1}) & = -\num{1,75}(\num{3,1}) + \num{1,75} = -\num{3,675} \\
    \end{split}
    \end{equation*}

    Iterasi Newton-Raphson: \\
    \begin{equation*}
        x_{i+1} = x_i - \frac{f(x_i)}{f'(x_i)}
    \end{equation*}

    Iterasi 1: \\
    \begin{equation*}
    \begin{split}
        x_{i+1} & = \num{3,1} - \frac{-\num{0,3587}}{-\num{3,675}} \\
        & = \num{3,0024} \\
        E_r & = \frac{\|3,0024 - 3,1\|}{3,0024} \times 100\% \\
        & = \num{3,25}\% \\
    \end{split}
    \end{equation*} \\

    Iterasi selanjutnya dirangkum dalam tabel berikut: 
    \begin{table}[h]
        \centering
        \begin{tabular}{cccc}
            \toprule
            Iterasi & $x_i$ & $x_{i+1}$ & Error Relatif (\%) \\
            \midrule
            1 & 3,1000 & 3,0024 & 3,25 \\
            2 & 3,0024 & 3,0000 & 0,08 \\
            \bottomrule
        \end{tabular}
        \label{tab:iterasi}
    \end{table} \\ \\

    Dari hasil iterasi, akar dari persamaan $f(x) = 0$ adalah: \\
    \begin{equation}
        x \approx 3,0000
    \end{equation}

    \item Gunakan metode Newton-Raphson untuk mendapatkan akar persamaan: \\
    \begin{equation*}
        f(x) = -\num{2,1} + \num{6,21}x - \num{3,9}x^2 + \num{0,667}x^3
    \end{equation*}
    \penyelesaian

    Diketahui: \\
    \begin{equation*}
    \begin{split}
        x_i & = 0 \\
        f(x) & = -\num{2,1} + \num{6,21}x - \num{3,9}x^2 + \num{0,667}x^3 \\
        f'(x) & = \num{6,21} - \num{7,8}x + \num{2,001}x^2
    \end{split}
    \end{equation*}

    Nilai Awal: \\
    \begin{equation*}
    \begin{split}
        f(\num{0}) & = -\num{2,1} + \num{6,21}(0) - \num{3,9}(0)^2 + \num{0,667}(0)^3 = -\num{2,1} \\
        f'(\num{0}) & = \num{6,21} - \num{7,8}(0) + \num{2,001}(0)^2 = \num{6,21}
    \end{split}
    \end{equation*}

    Iterasi Newton-Raphson:
    \begin{equation*}
        x_{i+1} = x_i - \frac{f(x_i)}{f'(x_i)}
    \end{equation*}

    Iterasi 1: \\
    \begin{equation*}
    \begin{split}
        x_{i+1} & = 0 - \frac{-\num{2,1}}{\num{6,21}} = \num{0,3381} \\
        E_r & = \frac{|0,3381 - 0|}{0,3381} \times 100\% = \num{100}\%
    \end{split}
    \end{equation*}

    Iterasi selanjutnya dirangkum dalam tabel berikut:

    \begin{table}[h]
        \centering
        \begin{tabular}{cccc}
            \toprule
            Iterasi & $x_i$ & $x_{i+1}$ & Error Relatif (\%) \\
            \midrule
            1 & 0,0000 & 0,3381 & 100,00 \\
            2 & 0,3381 & 0,5542 & 38,98 \\
            3 & 0,5542 & 0,6934 & 20,10 \\
            4 & 0,6934 & 0,7436 & 6,76 \\
            5 & 0,7436 & 0,7472 & 0,48 \\
            \bottomrule
        \end{tabular}
        \label{tab:iterasi2}
    \end{table}

    Dari hasil iterasi, akar dari persamaan $f(x) = 0$ adalah:
    \begin{equation}
        x \approx 0,7472
    \end{equation}


    \item Gunakan metode Newton-Raphson untuk mendapatkan akar persamaan: \\
    \begin{equation*}
        f(x) = -\num{23,33} + \num{79,35}x - \num{88,09}x^2 + \num{41,6}x^3 - \num{8,68}x^4 + \num{0,658}x^5
    \end{equation*}
    dengan $x_i = \num{3,5}$ \\
    \penyelesaian

    Diketahui: \\
    \begin{equation*}
    \begin{split}
        x_i & = \num{3,5} \\
        f(x) & = -\num{23,33} + \num{79,35}x - \num{88,09}x^2 + \num{41,6}x^3 - \num{8,68}x^4 + \num{0,658}x^5 \\
        f'(x) & = \num{79,35} - \num{176,18}x + \num{124,8}x^2 - \num{34,72}x^3 + \num{3,29}x^4
    \end{split}
    \end{equation*}
    
    Nilai Awal: \\
    \begin{equation*}
    \begin{split}
        f(\num{3,5}) & = -\num{23,33} + \num{79,35}(3,5) - \num{88,09}(3,5)^2 + \num{41,6}(3,5)^3 - \num{8,68}(3,5)^4 + \num{0,658}(3,5)^5 \approx -\num{3,7975} \\
        f'(\num{3,5}) & = \num{79,35} - \num{176,18}(3,5) + \num{124,8}(3,5)^2 - \num{34,72}(3,5)^3 + \num{3,29}(3,5)^4 \approx -\num{20,9435}
    \end{split}
    \end{equation*}
    
    Iterasi Newton-Raphson:
    \begin{equation*}
        x_{i+1} = x_i - \frac{f(x_i)}{f'(x_i)}
    \end{equation*}
    
    \begin{equation*}
    \begin{split}
        x_{i+1} & = 3,5 - \frac{-\num{3,7975}}{-\num{20,9435}} \approx 3,3187 \\
        E_r & = \frac{|\num{3,3187} - \num{3,5}|}{\num{3,3187}} \times 100\% = \num{5,46}\%
    \end{split}
    \end{equation*}
    
    Iterasi selanjutnya dirangkum dalam tabel berikut:
    \begin{table}[h]
        \centering
        \begin{tabular}{cccc}
            \toprule
            Iterasi & $x_i$ & $x_{i+1}$ & Error Relatif (\%) \\
            \midrule
            1 & 3,5000 & 3,3187 & 5,46 \\
            2 & 3,3187 & 3,2704 & 1,46 \\
            3 & 3,2704 & 3,2689 & 0,05 \\
            \bottomrule
        \end{tabular}
        \caption{Hasil iterasi Newton-Raphson untuk soal nomor 3}
        \label{tab:iterasi3}
    \end{table}
    
    Dari hasil iterasi, akar dari persamaan $f(x) = 0$ adalah:
    \begin{equation}
        x \approx 3,2689
    \end{equation}

    \item Gunakan metode Secant untuk mendapatkan akar dari persamaan:
    \begin{equation*}
        f(x) = \num{9,36} - \num{21,963}x + \num{16,2965}x^2 - \num{3,70377}x^3
    \end{equation*}
    \penyelesaian Dengan $x_{i-1} = 0$ dan $x_i = 1$, maka: \\
    $f(x_{i-1}) = f(0) = \num{9,36} - \num{21,963}(0) + \num{16,2965}(0)^2 - \num{3,70377}(0)^3 = \num{9,36}$ dan \\ 
    $f(x_{i}) = f(1) = \num{9,36} - \num{21,963}(1) + \num{16,2965}(1)^2 - \num{3,70377}(1)^3 = \num{-0,01027}$. \\

    Menggunakan metode Secant, dapat diperoleh hasil iterasi pertama:
    \begin{align*}
        x_{i+1} &= x_i - \frac{f(x_i)(x_{i-1} - x_i)}{f(x_{i-1}) - f(x_i)} \\
        &= 1 - \frac{(-\num{0,01027})(-1)}{\num{9,36} - \num{0,01027}} \\
        &= \num{0,99890}
    \end{align*}

    Dengan iterasi berikutnya hingga nilai $|f(x_i)|$ mendekati nol, diperoleh hasil sebagai berikut.\\
    \begin{tabular}{|c|c|c|c|c|c|}
        \hline
        iterasi & $x_{i-1}$ & $x_i$ & $f(x_{i-1})$ & $f(x_i)$ & $x_{i+1}$ \\
        \hline
        1 & \num{0,00000} & \num{1,00000} & \num{9,36000} & \num{-0,01027} & \num{0,99890}\\
        2 & \num{1,00000} & \num{0,99890} & \num{-0,01027} & \num{-0,00974} & \num{0,97891}\\
        3 & \num{0,99890} & \num{0,97891} & \num{-0,00974} & \num{0,00222} & \num{0,98262}\\
        4 & \num{0,97891} & \num{0,98262} & \num{0,00222} & \num{-0,00032} & \num{0,98215}\\
        5 & \num{0,98262} & \num{0,98215} & \num{-0,00032} & \num{-0,00001} & \num{0,98214}\\
        6 & \num{0,98215} & \num{0,98214} & \num{-0,00001} & \num{0,00000} & \num{0,98214}\\
         \hline
        \end{tabular} \\
    Dengan demikian, akar dari $f(x)$ adalah $x \approx \num{0,98214}$.

    \item Gunakan metode Secant untuk mendapatkan akar dari persamaan:
    \begin{equation*}
        f(x) = x^4 - \num{8,6}x^3 - \num{35,51}x^2 + \num{464}x - \num{998,46}
    \end{equation*}
    dengan $x_{i-1} = \num{7}$ dan $x_i = \num{9}$ \\
    \penyelesaian Dengan \\
    $f(x_{i-1}) = f(7) = 7^4 - \num{8,6}(7)^3 - \num{35,51}(7)^2 + \num{464}(7) - \num{998,46} = \num{-39,25}$ dan \\
    $f(x_i) = f(9) = 9^4 - \num{8,6}(9)^3 - \num{35,51}(9)^2 + \num{464}(9) - \num{998,46} = \num{592,83}$, \\
    Menggunakan metode Secant, dapat diperoleh hasil iterasi pertama:
    \begin{align*}
        x_{i+1} &= x_i - \frac{f(x_i)(x_{i-1} - x_i)}{f(x_{i-1}) - f(x_i)} \\
        &= 9 - \frac{(\num{529,83})(7 - 9)}{\num{-39,25} - \num{592,83}} \\
        &= \num{7,12419}
    \end{align*}

    Dengan iterasi berikutnya hingga nilai $|f(x_i)|$ mendekati nol, diperoleh hasil sebagai berikut.\\
    \begin{tabular}{|c|c|c|c|c|c|}
        \hline
        iterasi & $x_{i-1}$ & $x_i$ & $f(x_{i-1})$ & $f(x_i)$ & $x_{i+1}$ \\
        \hline
        1 & \num{7,00000} & \num{9,00000} & \num{-39,25000} & \num{592,83000} & \num{7,12419}\\
        2 & \num{9,00000} & \num{7,12419} & \num{592,83000} & \num{-28,73897} & \num{7,21092}\\
        3 & \num{7,12419} & \num{7,21092} & \num{-28,73897} & \num{-19,85323} & \num{7,40470}\\
        4 & \num{7,21092} & \num{7,40470} & \num{-19,85323} & \num{5,03497} & \num{7,36550}\\
        5 & \num{7,40470} & \num{7,36550} & \num{5,03497} & \num{-0,59129} & \num{7,36962}\\
        6 & \num{7,36550} & \num{7,36962} & \num{-0,59129} & \num{-0,01458} & \num{7,36972}\\
        7 & \num{7,36962} & \num{7,36972} & \num{-0,01458} & \num{0,00004} & \num{7,36972}\\
        8 & \num{7,36972} & \num{7,36972} & \num{0,00004} & \num{-0,00000} & \num{7,36972}\\
         \hline
        \end{tabular} \\
    Dengan demikian, akar dari $f(x)$ adalah $x \approx \num{7,36972}$.

    \item Gunakan metode Secant untuk mendapatkan akar dari persamaan:
    \begin{equation*}
        f(x) = x^3 - \num{6}x^2 + \num{11}x - \num{6}
    \end{equation*}
    dengan $x_{i-1} = \num{2,5}$ dan $x_i = \num{3,6}$ \\
    \penyelesaian Dengan \\
    $f(x_{i-1}) = f(2,5) = (2,5)^3 - \num{6}(2,5)^2 + \num{11}(2,5) - \num{6} = \num{-0,375}$ dan \\
    $f(x_i) = f(3,6) = (3,6)^3 - \num{6}(3,6)^2 + \num{11}(3,6) - \num{6} = \num{2,496}$. \\
    Menggunakan metode Secant, dapat diperoleh hasil iterasi pertama:
    \begin{align*}
        x_{i+1} &= x_i - \frac{f(x_i)(x_{i-1} - x_i)}{f(x_{i-1}) - f(x_i)} \\
        &= \num{3,6} - \frac{(\num{2,496})(\num{2,5} - \num{3,6})}{\num{-0,375} - \num{2,496}} \\
        &= \num{2,64368}.
    \end{align*}

    Dengan iterasi berikutnya hingga nilai $|f(x_i)|$ mendekati nol, diperoleh hasil sebagai berikut.\\
    \begin{tabular}{|c|c|c|c|c|c|}
        \hline
        iterasi & $x_{i-1}$ & $x_i$ & $f(x_{i-1})$ & $f(x_i)$ & $x_{i+1}$ \\
        \hline
        1 & \num{2,50000} & \num{3,60000} & \num{-0,37500} & \num{2,49600} & \num{2,64368}\\
        2 & \num{3,60000} & \num{2,64368} & \num{2,49600} & \num{-0,37699} & \num{2,76917}\\
        3 & \num{2,64368} & \num{2,76917} & \num{-0,37699} & \num{-0,31412} & \num{3,39610}\\
        4 & \num{2,76917} & \num{3,39610} & \num{-0,31412} & \num{1,32505} & \num{2,88931}\\
        5 & \num{3,39610} & \num{2,88931} & \num{1,32505} & \num{-0,18598} & \num{2,95169}\\
        6 & \num{2,88931} & \num{2,95169} & \num{-0,18598} & \num{-0,08974} & \num{3,00985}\\
        7 & \num{2,95169} & \num{3,00985} & \num{-0,08974} & \num{0,01999} & \num{2,99925}\\
        8 & \num{3,00985} & \num{2,99925} & \num{0,01999} & \num{-0,00149} & \num{2,99999}\\
        9 & \num{2,99925} & \num{2,99999} & \num{-0,00149} & \num{-0,00002} & \num{3,00000}\\
        10 & \num{2,99999} & \num{3,00000} & \num{-0,00002} & \num{0,00000} & \num{3,00000}\\
         \hline
        \end{tabular} \\
    Dengan demikian, akar dari $f(x)$ adalah $x \approx 3$.

    \item Buatlah sebuah paparan untuk menjelaskan tentang metode Bairstow dan metode Quotient-Difference (Q-D). 
    Dan buatlah sebuah kesimpulan mengenai kemudahan/kesulitan kedua metode tersebut didalam menyelesaikan masalah dibanding dengan metode2 yang telah anda pelajari dalam materi ini. \\
    \penyelesaian
\end{enumerate}

\end{document}