\documentclass{article}
\usepackage[utf8]{inputenc}
\usepackage[a4paper, margin=1in]{geometry}
\usepackage{siunitx}
\usepackage{amsmath}
\usepackage{enumitem}
\usepackage{esdiff}
\usepackage{pgfplots}
\usepackage{listings}
\usepackage{xcolor}

\pgfplotsset{compat=1.18, width=10cm}

\tolerance=1
\emergencystretch=\maxdimen
\hyphenpenalty=10000
\hbadness=10000

\sisetup{
    input-ignore={.},
    output-decimal-marker={,},
    group-minimum-digits=4,
    group-separator={.},
    group-digits=integer
}

\definecolor{darkgray}{rgb}{.4,.4,.4}

\lstdefinestyle{code}{
    aboveskip={1.3\baselineskip},
    basicstyle=\normalsize\ttfamily\linespread{4},
    breaklines=false,
    columns=fullflexible,
    commentstyle=\color[rgb]{0.127,0.427,0.514}\ttfamily\itshape,
    escapechar=@,
    extendedchars=true,
    frame=single,
    identifierstyle=\color{black},
    inputencoding=latin1,
    keywordstyle=\color[HTML]{228B22}\bfseries,
    ndkeywordstyle=\color[HTML]{228B22}\bfseries,
    numbers=left,
    numberstyle=\normalsize,
    prebreak = \raisebox{0ex}[0ex][0ex]{\ensuremath{\hookleftarrow}},
    stringstyle=\color[rgb]{0.639,0.082,0.082}\ttfamily,
    upquote=true,
    showstringspaces=false,
    xleftmargin=5ex,
    aboveskip=5pt
}

\newcommand{\penyelesaian}{\textbf{Penyelesaian: }}

\title{\textbf{Komputasi Numerik: Tugas 3}}
\author{Kelompok 15}
\date{}

\begin{document}

\maketitle

\begin{enumerate}
    \item Dapatkan akar-akar persamaan berikut: 
    \begin{enumerate}
        \item $x^3 + \num{6,6}x^2 - \num{29,05}x + \num{22,64} = 0$
        \item $x^4 - \num{0,41}x^3 + \num{1,632}x^2 - \num{9,146}x + \num{7,260} = 0$
    \end{enumerate}
    Dengan metode Iterasi. \\
    \penyelesaian Penyelesaian soal 1

    \item Dapatkan akar-akar persamaan berikut: 
    \begin{enumerate}
        \item $x^3 + \num{6,6}x^2 - \num{29,05}x + \num{22,64} = 0$
        \item $x^4 - \num{0,41}x^3 + \num{1,632}x^2 - \num{9,146}x + \num{7,260} = 0$
    \end{enumerate}
    Dengan metode Faktorisasi. \\
    \penyelesaian Penyelesaian soal 2

    \item Gunakan metode Newton-Raphson untuk mendapatkan akar persamaan: \\
    \begin{equation*}
        f(x) = -\num{0,875}x^2 + \num{1,75}x + \num{2,625}
    \end{equation*}
    dengan $x_i = \num{3,1}$ \\
    \penyelesaian

    \item Gunakan metode Newton-Raphson untuk mendapatkan akar persamaan: \\
    \begin{equation*}
        f(x) = -\num{2,1} + \num{6,21}x - \num{3,9}x^2 + \num{0,667}x^3
    \end{equation*}
    \penyelesaian

    \item Gunakan metode Newton-Raphson untuk mendapatkan akar persamaan: \\
    \begin{equation*}
        f(x) = -\num{23,33} + \num{79,35}x - \num{88,09}x^2 + \num{41,6}x^3 - \num{8,68}x^4 + \num{0,658}x^5
    \end{equation*}
    dengan $x_i = \num{3,5}$ \\
    \penyelesaian

    \item Gunakan metode Secant untuk mendapatkan akar dari persamaan:
    \begin{equation*}
        f(x) = \num{9,36} - \num{21,963}x + \num{16,2965}x^2 - \num{3,70377}x^3
    \end{equation*}
    \penyelesaian

    \item Gunakan metode Secant untuk mendapatkan akar dari persamaan:
    \begin{equation*}
        f(x) = x^4 - \num{8,6}x^3 - \num{35,51}x^2 + \num{464}x - \num{998,46}
    \end{equation*}
    dengan $x_{i-1} = \num{7}$ dan $x_i = \num{9}$ \\
    \penyelesaian

    \item Gunakan metode Secant untuk mendapatkan akar dari persamaan:
    \begin{equation*}
        f(x) = x^3 - \num{6}x^2 + \num{11}x - \num{6}
    \end{equation*}
    dengan $x_{i-1} = \num{2,5}$ dan $x_i = \num{3,6}$ \\
    \penyelesaian

    \item Buatlah sebuah paparan untuk menjelaskan tentang metode Bairstow dan metode Quotient-Difference (Q-D). 
    Dan buatlah sebuah kesimpulan mengenai kemudahan/kesulitan kedua metode tersebut didalam menyelesaikan masalah dibanding dengan metode2 yang telah anda pelajari dalam materi ini. \\
    \penyelesaian
    \begin{itemize}
        \item Metode Bairstow \\
        Metode Bairstow adalah metode numerik untuk mendapatkan akar polinomial, dapat berupa akar real maupun akar komples.
        Metode ini menggunakan pendekatan iteratif dengan dua parameter $r$ dan $s$ dimana kedua parameter tersebut berkaitan dengan faktor kuadratik $x^2 + rx + s$ dari polinomial tersebut.
        Pendekatan iteratif dilakukan dengan cara melakukan pembagian polinomial dan hasil pembagiannya terhadap $x^2 + rx + s$ terus menerus hingga sisa pembagian konvergen terhadap nol, dimana faktor-faktor kuadratiknya merupakan faktor paling sederhana.
        \item Metode Quotient-Difference (Q-D) \\
        Metode Q-D adalah metode numerik untuk mendapatkan akar polinomial yang lebih efisien, karena tidak memerlukan tebakan awal seperti metode-metode lain.
        Metode ini menggunakan pendekatan iteratif dengan mengandalkan perhitungan rasio (quotient) dan selisih (difference) dari hasil pembagian antara polinomial dan faktornya.
        Polinomial yang ingin dicari akar-akarnya akan dibagi dengan faktor linier, lalu selisih dan rasio dari pembagian tersebut dihitung untuk mendapatkan akar.
        Proses ini dilakukan secara berulang-ulang hingga akar konvergen, yaitu ketika nilai-nilainya tidak berubah secara signifikan.
    \end{itemize}

    \textbf{Kesimpulan: } Dibandingkan dengan metode-metode lain kedua metode ini lebih mudah dilakukan oleh komputer karena lebih sistematis.
    Namun, metode-metode lain masih tetap lebih mudah dilakukan. Misalnya Newton-Raphson dan Secant sangat efisien untuk menemukan akar tunggal, metode faktorisasi lebih baik jika faktor-faktor polinomial dapat ditemukan dengan mudah. Dari semua metode yang telah disebutkan, metode iterasi paling tidak efisien tetapi paling mudah untuk diterapkan.

\end{enumerate}

\end{document}