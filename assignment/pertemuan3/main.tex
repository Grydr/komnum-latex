\documentclass{article}
\usepackage[utf8]{inputenc}
\usepackage[a4paper, margin=1in]{geometry}
\usepackage{siunitx}
\usepackage{amsmath}
\usepackage{enumitem}
\usepackage{esdiff}
\usepackage{pgfplots}
\usepackage{listings}
\usepackage{xcolor}
\usepackage{booktabs}

\pgfplotsset{compat=1.18, width=10cm}

\tolerance=1
\emergencystretch=\maxdimen
\hyphenpenalty=10000
\hbadness=10000

\sisetup{
    input-ignore={.},
    output-decimal-marker={,},
    group-minimum-digits=4,
    group-separator={.},
    group-digits=integer
}

\definecolor{darkgray}{rgb}{.4,.4,.4}

\lstdefinestyle{code}{
    aboveskip={1.3\baselineskip},
    basicstyle=\normalsize\ttfamily\linespread{4},
    breaklines=false,
    columns=fullflexible,
    commentstyle=\color[rgb]{0.127,0.427,0.514}\ttfamily\itshape,
    escapechar=@,
    extendedchars=true,
    frame=single,
    identifierstyle=\color{black},
    inputencoding=latin1,
    keywordstyle=\color[HTML]{228B22}\bfseries,
    ndkeywordstyle=\color[HTML]{228B22}\bfseries,
    numbers=left,
    numberstyle=\normalsize,
    prebreak = \raisebox{0ex}[0ex][0ex]{\ensuremath{\hookleftarrow}},
    stringstyle=\color[rgb]{0.639,0.082,0.082}\ttfamily,
    upquote=true,
    showstringspaces=false,
    xleftmargin=5ex,
    aboveskip=5pt
}

\newcommand{\penyelesaian}{\textbf{Penyelesaian: }}

\title{\textbf{Komputasi Numerik: Tugas 3}}
\author{Kelompok 15}
\date{}

\begin{document}

\maketitle

\begin{enumerate}
    \item Dapatkan akar-akar persamaan berikut: 
    \begin{enumerate}
        \item $x^3 + \num{6,6}x^2 - \num{29,05}x + \num{22,64} = 0$
        \item $x^4 - \num{0,41}x^3 + \num{1,632}x^2 - \num{9,146}x + \num{7,260} = 0$
    \end{enumerate}
    Dengan metode Iterasi. \\
    \penyelesaian Penyelesaian soal 1

    \item Dapatkan akar-akar persamaan berikut: 
    \begin{enumerate}
        \item $x^3 + \num{6,6}x^2 - \num{29,05}x + \num{22,64} = 0$
        \item $x^4 - \num{0,41}x^3 + \num{1,632}x^2 - \num{9,146}x + \num{7,260} = 0$
    \end{enumerate}
    Dengan metode Faktorisasi. \\
    \penyelesaian Penyelesaian soal 2

    \item Gunakan metode Newton-Raphson untuk mendapatkan akar persamaan: \\
    \begin{equation*}
        f(x) = -\num{0,875}x^2 + \num{1,75}x + \num{2,625}
    \end{equation*}
    dengan $x_i = \num{3,1}$ \\
    \penyelesaian

    Diketahui: \\
    \begin{equation*}
    \begin{split}
        x_i & = \num{3,1} \\
        f(x) & = -\num{0,875}x^2 + \num{1,75}x + \num{2,625} \\
        f'(x) & = -\num{1,75}x + \num{1,75} \\ 
    \end{split}
    \end{equation*}

    Nilai awal: \\
    \begin{equation*}
    \begin{split}
        f(\num{3,1}) & = -\num{0,875}(\num{3,1})^2 + \num{1,75}(\num{3,1}) + \num{2,625} = -\num{0,35875} \\
        f'(\num{3,1}) & = -\num{1,75}(\num{3,1}) + \num{1,75} = -\num{3,675} \\
    \end{split}
    \end{equation*}

    Iterasi Newton-Raphson: \\
    \begin{equation*}
        x_{i+1} = x_i - \frac{f(x_i)}{f'(x_i)}
    \end{equation*}

    Iterasi 1: \\
    \begin{equation*}
    \begin{split}
        x_{i+1} & = \num{3,1} - \frac{-\num{0,3587}}{-\num{3,675}} \\
        & = \num{3,0024} \\
        E_r & = \frac{\|3,0024 - 3,1\|}{3,0024} \times 100\% \\
        & = \num{3,25}\% \\
    \end{split}
    \end{equation*} \\

    Iterasi selanjutnya dirangkum dalam tabel berikut: 
    \begin{table}[h]
        \centering
        \begin{tabular}{cccc}
            \toprule
            Iterasi & $x_i$ & $x_{i+1}$ & Error Relatif (\%) \\
            \midrule
            1 & 3,1000 & 3,0024 & 3,25 \\
            2 & 3,0024 & 3,0000 & 0,08 \\
            \bottomrule
        \end{tabular}
        \label{tab:iterasi}
    \end{table} \\ \\

    Dari hasil iterasi, akar dari persamaan $f(x) = 0$ adalah: \\
    \begin{equation}
        x \approx 3,0000
    \end{equation}

    \item Gunakan metode Newton-Raphson untuk mendapatkan akar persamaan: \\
    \begin{equation*}
        f(x) = -\num{2,1} + \num{6,21}x - \num{3,9}x^2 + \num{0,667}x^3
    \end{equation*}
    \penyelesaian

    Diketahui: \\
    \begin{equation*}
    \begin{split}
        x_i & = 0 \\
        f(x) & = -\num{2,1} + \num{6,21}x - \num{3,9}x^2 + \num{0,667}x^3 \\
        f'(x) & = \num{6,21} - \num{7,8}x + \num{2,001}x^2
    \end{split}
    \end{equation*}

    Nilai Awal: \\
    \begin{equation*}
    \begin{split}
        f(\num{0}) & = -\num{2,1} + \num{6,21}(0) - \num{3,9}(0)^2 + \num{0,667}(0)^3 = -\num{2,1} \\
        f'(\num{0}) & = \num{6,21} - \num{7,8}(0) + \num{2,001}(0)^2 = \num{6,21}
    \end{split}
    \end{equation*}

    Iterasi Newton-Raphson:
    \begin{equation*}
        x_{i+1} = x_i - \frac{f(x_i)}{f'(x_i)}
    \end{equation*}

    Iterasi 1: \\
    \begin{equation*}
    \begin{split}
        x_{i+1} & = 0 - \frac{-\num{2,1}}{\num{6,21}} = \num{0,3381} \\
        E_r & = \frac{|0,3381 - 0|}{0,3381} \times 100\% = \num{100}\%
    \end{split}
    \end{equation*}

    Iterasi selanjutnya dirangkum dalam tabel berikut:

    \begin{table}[h]
        \centering
        \begin{tabular}{cccc}
            \toprule
            Iterasi & $x_i$ & $x_{i+1}$ & Error Relatif (\%) \\
            \midrule
            1 & 0,0000 & 0,3381 & 100,00 \\
            2 & 0,3381 & 0,5542 & 38,98 \\
            3 & 0,5542 & 0,6934 & 20,10 \\
            4 & 0,6934 & 0,7436 & 6,76 \\
            5 & 0,7436 & 0,7472 & 0,48 \\
            \bottomrule
        \end{tabular}
        \label{tab:iterasi2}
    \end{table}

    Dari hasil iterasi, akar dari persamaan $f(x) = 0$ adalah:
    \begin{equation}
        x \approx 0,7472
    \end{equation}


    \item Gunakan metode Newton-Raphson untuk mendapatkan akar persamaan: \\
    \begin{equation*}
        f(x) = -\num{23,33} + \num{79,35}x - \num{88,09}x^2 + \num{41,6}x^3 - \num{8,68}x^4 + \num{0,658}x^5
    \end{equation*}
    dengan $x_i = \num{3,5}$ \\
    \penyelesaian

    Diketahui: \\
    \begin{equation*}
    \begin{split}
        x_i & = \num{3,5} \\
        f(x) & = -\num{23,33} + \num{79,35}x - \num{88,09}x^2 + \num{41,6}x^3 - \num{8,68}x^4 + \num{0,658}x^5 \\
        f'(x) & = \num{79,35} - \num{176,18}x + \num{124,8}x^2 - \num{34,72}x^3 + \num{3,29}x^4
    \end{split}
    \end{equation*}
    
    Nilai Awal: \\
    \begin{equation*}
    \begin{split}
        f(\num{3,5}) & = -\num{23,33} + \num{79,35}(3,5) - \num{88,09}(3,5)^2 + \num{41,6}(3,5)^3 - \num{8,68}(3,5)^4 + \num{0,658}(3,5)^5 \approx -\num{3,7975} \\
        f'(\num{3,5}) & = \num{79,35} - \num{176,18}(3,5) + \num{124,8}(3,5)^2 - \num{34,72}(3,5)^3 + \num{3,29}(3,5)^4 \approx -\num{20,9435}
    \end{split}
    \end{equation*}
    
    Iterasi Newton-Raphson:
    \begin{equation*}
        x_{i+1} = x_i - \frac{f(x_i)}{f'(x_i)}
    \end{equation*}
    
    \begin{equation*}
    \begin{split}
        x_{i+1} & = 3,5 - \frac{-\num{3,7975}}{-\num{20,9435}} \approx 3,3187 \\
        E_r & = \frac{|\num{3,3187} - \num{3,5}|}{\num{3,3187}} \times 100\% = \num{5,46}\%
    \end{split}
    \end{equation*}
    
    Iterasi selanjutnya dirangkum dalam tabel berikut:
    \begin{table}[h]
        \centering
        \begin{tabular}{cccc}
            \toprule
            Iterasi & $x_i$ & $x_{i+1}$ & Error Relatif (\%) \\
            \midrule
            1 & 3,5000 & 3,3187 & 5,46 \\
            2 & 3,3187 & 3,2704 & 1,46 \\
            3 & 3,2704 & 3,2689 & 0,05 \\
            \bottomrule
        \end{tabular}
        \caption{Hasil iterasi Newton-Raphson untuk soal nomor 3}
        \label{tab:iterasi3}
    \end{table}
    
    Dari hasil iterasi, akar dari persamaan $f(x) = 0$ adalah:
    \begin{equation}
        x \approx 3,2689
    \end{equation}

    \item Gunakan metode Secant untuk mendapatkan akar dari persamaan:
    \begin{equation*}
        f(x) = \num{9,36} - \num{21,963}x + \num{16,2965}x^2 - \num{3,70377}x^3
    \end{equation*}
    \penyelesaian

    \item Gunakan metode Secant untuk mendapatkan akar dari persamaan:
    \begin{equation*}
        f(x) = x^4 - \num{8,6}x^3 - \num{35,51}x^2 + \num{464}x - \num{998,46}
    \end{equation*}
    dengan $x_{i-1} = \num{7}$ dan $x_i = \num{9}$ \\
    \penyelesaian

    \item Gunakan metode Secant untuk mendapatkan akar dari persamaan:
    \begin{equation*}
        f(x) = x^3 - \num{6}x^2 + \num{11}x - \num{6}
    \end{equation*}
    dengan $x_{i-1} = \num{2,5}$ dan $x_i = \num{3,6}$ \\
    \penyelesaian

    \item Buatlah sebuah paparan untuk menjelaskan tentang metode Bairstow dan metode Quotient-Difference (Q-D). 
    Dan buatlah sebuah kesimpulan mengenai kemudahan/kesulitan kedua metode tersebut didalam menyelesaikan masalah dibanding dengan metode2 yang telah anda pelajari dalam materi ini. \\
    \penyelesaian

\end{enumerate}

\end{document}