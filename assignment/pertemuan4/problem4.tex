\documentclass{article}
\usepackage{amsmath}

\begin{document}

\section*{Interpolasi Lagrange}

\subsection*{Data}
\[
\begin{array}{c|ccccc}
T & 361 & 367 & 378 & 387 & 399 \\
\hline
P & 154.9 & 167.0 & 191.0 & 212.5 & 244.2 \\
\end{array}
\]

\subsection*{Perhitungan \( P(372) \)}


\begin{align*}
P(372) = \;& 154.9 \cdot \frac{(372-367)(372-378)(372-387)(372-399)}{(361-367)(361-378)(361-387)(361-399)} \\
& + 167.0 \cdot \frac{(372-361)(372-378)(372-387)(372-399)}{(367-361)(367-378)(367-387)(367-399)} \\
& + 191.0 \cdot \frac{(372-361)(372-367)(372-387)(372-399)}{(378-361)(378-367)(378-387)(378-399)} \\
& + 212.5 \cdot \frac{(372-361)(372-367)(372-378)(372-399)}{(387-361)(387-367)(387-378)(387-399)} \\
& + 244.2 \cdot \frac{(372-361)(372-367)(372-378)(372-387)}{(399-361)(399-367)(399-378)(399-387)}
\end{align*}

\subsection*{Hasil}
\[
P(372) \approx 154.9 \times (-0.0436) + 167.0 \times 0.6316 + 191.0 \times 0.6689 + 212.5 \times (-0.1325) + 244.2 \times 0.0020
\]
\[
P(372) \approx -6.75 + 105.48 + 127.76 - 28.16 + 0.49 = \boxed{198.82}
\]

\end{document}