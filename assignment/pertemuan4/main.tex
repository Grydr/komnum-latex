\documentclass{article}
\usepackage[utf8]{inputenc}
\usepackage[a4paper, margin=1in]{geometry}
\usepackage{siunitx}
\usepackage{amsmath}
\usepackage{enumitem}
\usepackage{esdiff}
\usepackage{pgfplots}
\usepackage{listings}
\usepackage{xcolor}

\pgfplotsset{compat=1.18, width=10cm}

\tolerance=1
\emergencystretch=\maxdimen
\hyphenpenalty=10000
\hbadness=10000

\sisetup{
    input-ignore={.},
    output-decimal-marker={,},
    group-minimum-digits=4,
    group-separator={.},
    group-digits=integer
}

\definecolor{darkgray}{rgb}{.4,.4,.4}

\lstdefinestyle{code}{
    aboveskip={1.3\baselineskip},
    basicstyle=\normalsize\ttfamily\linespread{4},
    breaklines=false,
    columns=fullflexible,
    commentstyle=\color[rgb]{0.127,0.427,0.514}\ttfamily\itshape,
    escapechar=@,
    extendedchars=true,
    frame=single,
    identifierstyle=\color{black},
    inputencoding=latin1,
    keywordstyle=\color[HTML]{228B22}\bfseries,
    ndkeywordstyle=\color[HTML]{228B22}\bfseries,
    numbers=left,
    numberstyle=\normalsize,
    prebreak = \raisebox{0ex}[0ex][0ex]{\ensuremath{\hookleftarrow}},
    stringstyle=\color[rgb]{0.639,0.082,0.082}\ttfamily,
    upquote=true,
    showstringspaces=false,
    xleftmargin=5ex,
    aboveskip=5pt
}

\newcommand{\penyelesaian}{\textbf{Penyelesaian: }}

\title{\textbf{Komputasi Numerik: Tugas 4}}
\author{Kelompok 15}
\date{}

\begin{document}

\maketitle

\begin{enumerate}
    \item Carilah nilai $y(\num{2,1})$ menggunakan Interpolasi Linier, Kuadratik, NGF, dan Lagrange, jika diketahui data-data berikut: \\
    \begin{tabular}{ c c c c c c }
        x & 2 & 4 & 6 & 8 & 10 \\
        y & \num{9,68} & \num{10,96} & \num{12,32} & \num{13,76} & \num{15,28} \\
    \end{tabular} \\
    \penyelesaian

    \item Carilah nilai $e^{\num{2,00}}$ menggunakan Interpolasi NGB, jika diketahui data-data berikut: \\
    \begin{tabular}{ c c c c c c }
        x & \num{0,1} & \num{0,6} & \num{1,1} & \num{1,6} & \num{2,1} \\
        y & \num{1,1052} & \num{1,8221} & \num{3,0042} & \num{4,9530} & \num{8,1662} \\
    \end{tabular} \\
    \penyelesaian 

    \item Carilah nilai $\sin(\num{0,28})$ menggunakan Interpolasi Gauss Forward, Gauss Backward, dan Hermite, jika diketahui data-data berikut: \\
    \begin{tabular}{ c c c c c c }
        x & \num{0,15} & \num{0,20} & \num{0,25} & \num{0,30} & \num{0,35} \\
        y & \num{0,1494} & \num{0,1986} & \num{0,2474} & \num{0,2955} & \num{0,3429} \\
    \end{tabular} \\
    \penyelesaian

    \item Tentukan tekanan uap pada temperatur $372^{\circ}$, jika diketahui hubungan antara tekanan uap dan temperatur sebuah bejana adalah sebagai berikut: \\
        \begin{tabular}{ c c c c c c }
        T & $361^{\circ}$ & $367^{\circ}$ & $378^{\circ}$ & $387^{\circ}$ & $399^{\circ}$ \\
        P & \num{154,9} & \num{167,0} & \num{191,0} & \num{212,5} & \num{244,2} \\
    \end{tabular} \\
    \penyelesaian

    \item Fungsi interpolasi polynomial yang telah anda pelajari akan sulit digunakan apabila fungsi yang hendak diinterpolasikan memiliki banyak fluktuasi, sehingga akan memerlukan banyak titik data.
    Metode alternatif yang dapat digunakan adalah menerapkan fungsi interpolasi polynomial untuk bagian per bagian kurva. 
    Metode ini dikenal dengan sebutan Interpolasi Spline. Buatlah sebuah paparan untuk menjelaskan tentang Interpolasi Spline, yang meliputi: konsep dasar, Spline Linier, Spline Kuadratik, dan Spline Kubik. \\
    \penyelesaian

\end{enumerate}

\end{document}